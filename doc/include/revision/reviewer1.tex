\begin{reviewer}
\clabel{1}{1}\\
The main contribution of this paper is the proposal of a methodology for
considering process variation in power and temperature modeling that is
100--200$\times$ faster than Monte Carlo simulation. The motivation for
considering process variation is well stated, and the results indeed show that
deterministic modeling when used for optimization fails to meet performance
targets once process variation is taken into account.
\end{reviewer}

\begin{authors}
Thank you.
\end{authors}

\begin{reviewer}
\clabel{1}{2}\\
As a proof of concept the authors propose a simple optimization algorithm for
solving the scheduling problem. They show that their methodology is able to
achieve optimization within the given constraints whereas deterministic
optimization underestimates temperature and power such that constraints are not
met. Although this optimization methodology is necessary and much faster than
na\"{i}ve methods such as Monte Carlo simulation, the reviewer feel that the
scheduling problem is a poor example to use since task scheduling is a problem
that must be solved at run time not at design time.
\end{reviewer}

\begin{authors}
The reviewer is right that some problems require dynamic scheduling. In this
paper, however, we consider an optimization problem (Sec.~IX-A) that involves
static scheduling, which is typically done offline, that is, at design time.
Since our work is targeted at the design stage, we consider static scheduling
to be a prominent application area of the proposed techniques.

Let us also note that the importance of dynamic scheduling is well understood
and gives us a reach set of potential directions for future work.

\begin{actions}
  \action{The motivation behind static scheduling has been clarified in
  Sec.~IX-A.}
\end{actions}
\end{authors}

\begin{reviewer}
\clabel{1}{3}\\
Six hours to optimize the scheduling for a 30ms program is not reasonable.
\end{reviewer}

\begin{authors}
We agree that such a computational demand would not be acceptable for online
usage and might not be justifiable for certain classes of offline problems.
However, as noted earlier, our methodology has been developed with design-time
applications in mind, and, in that case, computational time plays an important
but secondary role. Even days might be readily acceptable if they can help to
decrease the maintenance costs of a product which is to serve well for years.
In our opinion, the reported optimization time is affordable, especially for
the industry.

Let us clarify the 30~ms mentioned by the reviewer. In our experiments, the
execution time of \emph{one} task is distributed between 10 and 30~ms, and the
number of tasks per application varies from 40 to 640 (Sec.~X-C). Therefore,
the actual application periods, which we apply our analysis to, are much longer
than 30~ms. It is also worth noting that the granularity of the analysis---in
particular, the sampling interval $\dt$ (Sec.~VII-C and Appendix~A)---is a
variable that the user is free to tune and set to the value that makes the most
sense to the problem at hand. As a result, an application with a period of one
second might be comparable in terms of optimization time with, for example, an
application with a period of one minute.

We would like to emphasize that the focal point of this work is the
mathematical foundation and basic structure of the proposed solutions, and the
corresponding implementation is merely a proof of concept, as pointed out by
the reviewer in \cref{1}{2}.

\begin{actions}
  \action{The confusion regarding the execution time of a task has been
  eliminated in Sec.~X-A.}
  \action{The number of tasks per application has been spelled out explicitly
  in Sec.~X-C.}
  \action{It has been noted in Sec.~X-A that $\dt$ might be adjusted when it is
  needed.}
\end{actions}
\end{authors}

\begin{reviewer}
\clabel{1}{4}\\
However the reviewer can recognize how this methodology would be quite useful
for design optimization. Perhaps a more realistic example could be used to
motivate the results.
\end{reviewer}

\begin{authors}
We agree with the reviewer.

\begin{actions}
  \action{One.}
\end{actions}
\end{authors}

\begin{reviewer}
\clabel{1}{5}\\
Furthermore, the process variation is considered at the processor core level,
but in reality process variation occurs at the transistor level. The reviewer
is wondering how this methodology would scale down to finer granularity, and
whether a design optimized at the core level would actually meat the design
requirements in real silicon, where process variation occurs at the transistor
level.
\end{reviewer}

\begin{authors}
\begin{actions}
  \action{One.}
\end{actions}
\end{authors}

\begin{reviewer}
\clabel{1}{6}\\
It is clear to the reviewer that this is an important and well developed
piece of work, and should be accepted for publication with minor revisions
addressing the two comments above.
\end{reviewer}

\begin{authors}
  Thank you for your appreciation.
\end{authors}
