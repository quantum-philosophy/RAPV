\begin{reviewer}
\clabel{1}{1}\\
The main contribution of this paper is the proposal of a methodology for
considering process variation in power and temperature modeling that is
100--200$\times$ faster than Monte Carlo simulation. The motivation for
considering process variation is well stated, and the results indeed show that
deterministic modeling when used for optimization fails to meet performance
targets once process variation is taken into account. As a proof of concept the
authors propose a simple optimization algorithm for solving the scheduling
problem. They show that their methodology is able to achieve optimization
within the given constraints whereas deterministic optimization underestimates
temperature and power such that constraints are not met. Although this
optimization methodology is necessary and much faster than na\"{i}ve methods
such as Monte Carlo simulation, the reviewer feel that the scheduling problem
is a poor example to use since task scheduling is a problem that must be solved
at run time not at design time. Six hours to optimize the scheduling for a 30ms
program is not reasonable. However the reviewer can recognize how this
methodology would be quite useful for design optimization. Perhaps a more
realistic example could be used to motivate the results. Furthermore, the
process variation is considered at the processor core level, but in reality
process variation occurs at the transistor level. The reviewer is wondering how
this methodology would scale down to finer granularity, and whether a design
optimized at the core level would actually meat the design requirements in real
silicon, where process variation occurs at the transistor level. It is clear to
the reviewer that this is an important and well developed piece of work, and
should be accepted for publication with minor revisions addressing the two
comments above.
\end{reviewer}

\begin{authors}
\begin{actions}
  \action{Action.}
\end{actions}
\end{authors}
