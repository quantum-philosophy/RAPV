\begin{reviewer}
\clabel{3}{1}
I think this paper has value in presenting use of various probabilistic schemes
for modeling and optimization.
\end{reviewer}

\begin{authors}
  Thank you.
\end{authors}

\begin{reviewer}
\clabel{3}{2}
However in order to be considered as a ``regular'' paper, it would have been
nice to get correlations between their reliability and thermal models with with
real detailed simulations (leakage enhanced hotspot, EM simulation studies
etc.).
\end{reviewer}

\begin{authors}
We would like to clarify that the thermal model used in the experimental results
\emph{is} the HotSpot thermal model [Skadron, 2004]. To elaborate, for each
problem, we feed into HotSpot a floorplan file and HotSpot's native
configuration file (\texttt{hotspot.config}), and HotSpot constructs an
equivalent thermal RC circuit for us, which is essentially a system of ordinary
differential equations given by a pair of matrices: a thermal capacitance matrix
$\mCth$ and a thermal conductance matrix $\mGth$. This system is then
transformed as it is described in Sec.~VII-B, which is an exact mathematical
operation. Furthermore, the HotSpot thermal model that we use has been enhanced
to account for the leakage power. As we write in Sec.~X-A, the leakage modeling
is based on SPICE simulations using the NanGate cell library configured
according to the high-performance 45-nm PTM.

Regarding reliability modeling, one of the most valuable, from our perspective,
aspects of our approach is that we keep the actual reliability model intact as a
pluggable component, which is noted in Sec.~VIII-B. In this way, the user can
take advantage of the state-of-the-art reliability models in a straightforward
manner. Consequently, our framework remains adequate as long as the reliability
model plugged into it is adequate. In the experimental results, we utilize a
well-established reliability model describing the thermal-cycling fatigue
[JEDEC, 2014], and, based on the experience from the literature, we consider
this model to be practical.

Finally, we note the approach based on Monte Carlo sampling, which is used for
the comparison in Sec.~X, does not perform any model order reduction and, hence,
does not compromise the resulting accuracy (see \cref{2}{2}). Consequently, we
conclude that the accuracy of the proposed framework has been assessed using
reliable, detailed simulations.

\begin{actions}
  \action{The use of the HotSpot thermal model has been noted in Sec.~X-A.}
  \action{The correlation between the framework's accuracy and the accuracy of
  reliability models has been mentioned in Sec.~VIII-B.}
\end{actions}
\end{authors}

\begin{reviewer}
\clabel{3}{3}
Another deficiency in my view is that while the paper deals with manufacturing
variations, it does not mention anything about modeling errors (thermal, power
etc.). Generating more detailed validation data would be very hard.
\end{reviewer}

\begin{authors}
We are in alignment with the reviewer that modeling errors are of immediate
importance. Depending on the quantity of interest, our framework is a
superposition of certain models, and any inaccuracy in those models can
eventually propagate to the output of our analysis. However, as probably many
other people, we would like to stand on the shoulders on giants such that we can
make progress by focusing on what matters the most to our problem and taking
everything else as a solid foundation for our research. To this end, we take the
state-of-the-art models from the literature and build on top of them. We
consciously try not to question the validity of those models: that work has
already been done and would only distract us from our main subject.

More concretely, our works begins with the assumption that a robust
deterministic simulator of the quantity of interest is available; this is what
is denoted by Algorithm~X throughout the manuscript (see Sec.~VI-C and, in
particular, Algorithm~1). Consequently, in the same vein as our reply to
\cref{3}{2}, the proposed framework is expected to perform well provided that it
is applied within the domain of the models composing it. For example, as we
reply to \cref{1}{3}, the thermal model that we use has been designed with
architectural studies in mind and has been validated within this particular area
[Skadron,~2004], and, therefore, out framework is targeted at that area as well,
which is drawn attention to in the paper.

We agree with the reviewer that the above-mentioned aspect of our work should be
more pronounced in the manuscript, and we have included in the revised version
several notes reminding the reader about the impact of modeling errors on the
performance of the proposed techniques.

\begin{actions}
  \action{Modeling errors related to Algorithm~X have been mentioned in
  Sec.~VI-C.}
  \action{Modeling errors related to reliability analysis have been mentioned in
  Sec.~VIII-B.}
\end{actions}
\end{authors}

\begin{reviewer}
\clabel{3}{4}
Hence I recommend the paper be accepted in a brief form (since the mathematical
contributions are interesting and relevant). In my view the paper is deficient
as a regular submission.
\end{reviewer}

\begin{authors}
Thank you for your appreciation of the mathematical contribution of this work.
Based on the feedback received from the reviewers, we let ourselves conclude
that the presentation given in the manuscript is, for the most part, clear and
understandable, which we are undoubtedly proud of. We believe that the
publication would not have attained this quality if we had not had all 14 pages
at our disposal. The format of a regular journal publication enables us to
convey the essence and usefulness of the proposed solutions in a coherent,
consistent way.

We hope that our answers to the previous comments and the corresponding
modifications in the manuscript have addressed the reviewer's concerns in a
satisfactory way. We kindly ask the reviewer to preserve the current format of
the manuscript and, therefore, to preserve the aforementioned quality of the
manuscript for future readers. Thank you.
\end{authors}
