\begin{reviewer}
\clabel{2}{1}\\
This work addressed a very challenging problem of temperature-induced failures
at the system level with considering process variations. The proposed framework
utilized several available stochastic approaches to establish their uncertainty
analysis engine, and used an optimization procedure to demonstrate its
efficiency. I think the authors did a good job on an important and practical
problem in circuit designs. The description is very clear and completely, the
paper is well organized. I only have a few comments about the experimental
results.
\end{reviewer}

\begin{authors}
  Thank you for the kind words.
\end{authors}

\begin{reviewer}
\noindent\clabel{2}{2}\\
1)The authors used their framework to perform the Monte Carlo
simulation for each sample, then, compared their results with it, and
calculated the error. However, in their framework, they performed the model
order reduction, and, by this way, might induce unavoidable error in the MC
simulation. Hence, the results in Table I might be adjusted.
\end{reviewer}

\begin{authors}
\begin{actions}
  \action{One.}
\end{actions}
\end{authors}

\begin{reviewer}
\clabel{2}{3}\\
\noindent 2)Could the authors report the values of q\char`_max in the
experimental results?
\end{reviewer}

\begin{authors}
\begin{actions}
  \action{One.}
\end{actions}
\end{authors}

\begin{reviewer}
\clabel{2}{4}\\
\noindent 3)Typo: In page 12, nc=5 should be lc=5.
\end{reviewer}

\begin{authors}
\begin{actions}
  \action{One.}
\end{actions}
\end{authors}
