\begin{reviewer}
\clabel{2}{1}
This work addressed a very challenging problem of temperature-induced failures
at the system level with considering process variations. The proposed framework
utilized several available stochastic approaches to establish their uncertainty
analysis engine, and used an optimization procedure to demonstrate its
efficiency. I think the authors did a good job on an important and practical
problem in circuit designs. The description is very clear and completely, the
paper is well organized. I only have a few comments about the experimental
results.
\end{reviewer}

\begin{authors}
  Thank you for the kind words.
\end{authors}

\begin{reviewer}
\clabel{2}{2}
1)The authors used their framework to perform the Monte Carlo
simulation for each sample, then, compared their results with it, and
calculated the error. However, in their framework, they performed the model
order reduction, and, by this way, might induce unavoidable error in the MC
simulation. Hence, the results in Table I might be adjusted.
\end{reviewer}

\begin{authors}
The reviewer is right that Monte Carlo (MC) simulations are used as a reference
point for evaluating the accuracy of the proposed framework, and the proposed
framework performs model order reduction. However, the MC-based approach does
not have any kind of model order reduction inside, which we emphasize in the
second paragraph of Sec.~X-B. In other words, the MC-based approach samples the
original model and, hence, does not compromise the resulting accuracy. This is
the reason for considering MC sampling to be a gold etalon for comparison.

\begin{actions}
  \action{The description of the MC-based approach has been improved in
  Sec.~X-B.}
\end{actions}
\end{authors}

\begin{reviewer}
\clabel{2}{3}
2)Could the authors report the values of $\q_\maximal$ in the
experimental results?
\end{reviewer}

\begin{authors}
We purposely did not report the values of $\q_\maximal$ in Sec.~X-C due to the
following two main reasons. First, the actual values of this parameter are,
from our perspective, irrelevant to the goal of the study in that section
(the second paragraph of Sec.~X-C). We compare the deterministic and
probabilistic approaches, and they both share the same values of $\q_\maximal$.
Second, we apply the optimization procedure to 50 different problems, and
$\q_\maximal$ is set individually for each of the problems in order to ensure
that $\q_\maximal$ makes sense for the subsequent optimization. Hence, there
are 50 different values that $\q_\maximal$ takes, which might be unnecessarily
overwhelming for the reader.

We agree with the reviewer that more intuition about $\q_\maximal$ is needed,
and we have decided to report the temperature range that $\q_\maximal$ belongs
to in our experiments, which is 90--120${}^\circ{}C$.

\begin{actions}
  \action{The range of $\q_\maximal$ has been reported in Sec.~X-C.}
\end{actions}
\end{authors}

\begin{reviewer}
\clabel{2}{4}
3)Typo: In page 12, nc=5 should be lc=5.
\end{reviewer}

\begin{authors}
  Thank you for the careful observation.

\begin{actions}
  \action{The typo has been fixed.}
\end{actions}
\end{authors}
