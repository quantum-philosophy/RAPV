The key building block of our solutions developed in \sref{temperature-analysis} and \sref{reliability-analysis} is the uncertainty quantification technique presented in this section.
The main task of this technique is the propagation of uncertainty through the system, that is, from a set of inputs to a set of outputs.
Specifically, the inputs are the uncertain parameters $\vu$, and the outputs are the quantities that we are interested in studying.
The latter can be, for instance, the energy consumption, maximal temperature, or temperature profile of the system over a certain period of time.

Due to the inherent complexity, uncertainty quantification problems are typically viewed as approximation problems: one constructs a computationally efficient surrogate for the stochastic model under consideration first and then studies this light representation instead.
In order to construct such an approximation, we appeal to spectral methods \cite{maitre2010, janson1997, eldred2008}.
The approach taken in this work can be classified as non-intrusive since the system is treated as a ``black box.''

\subsection{Uncertainty Model}
Before we proceed to the construction, let us first refine our definition of $\vu = (\u_i)_{i = 1}^\nparams$.
Each $\u_i$ is a characteristic of a single transistor (consider, for example, the effective channel length), and, therefore, each device in the electrical circuits at hand can potentially have a different value of this parameter as, in general, the variability due to process variation is not uniform.
Consequently, each $\u_i$ can be viewed as a random process $\u_i: \outcomes \times \domain \to \real$ defined on an appropriate spatial domain $\domain \subset \real^3$.
Since this work is system-level oriented, we model each processing element with one variable for each such random process.
More specifically, we let $\u_{ij} = \u_i(\cdot, \vr_j)$ be the \rv\ representing the $i$th uncertain parameter at the $j$th processing element where $\vr_j$ stands for the spatial location of the processing element (\eg, the center of mass).
Thus, we redefine the parameterization of the problem as
\begin{equation} \elab{uncertain-parameters}
  \vu = (\u_i)_{i = 1}^{\nparams \nprocs}
\end{equation}
such that there is a one-to-one correspondence between $\u_i$, $i = 1, \dots, \nparams \nprocs$, and $\u_{ij}$, $i = 1, \dots, \nparams$, $j = 1, \dots, \nprocs$.
For instance, in our illustrative application with two process parameters, the total number of stochastic dimensions is $2 \nprocs$.
\begin{remark}
Some authors prefer to split the variability of a process parameter at a spatial location into several parts such as wafer-to-wafer, die-to-die, and within-die; see, \eg, \cite{juan2012}.
However, from the mathematical standpoint, it is sufficient to consider just one random variable per location which is adequately correlated with the other locations of interest.
\end{remark}

A description of $\vu$ is an input to our analysis given by the user, and we consider it to be a part of the system specification $\specification$.
A proper (complete, unambiguous) way to describe a set of \rvs\ is to specify their joint probability distribution function.
In practice, however, such exhaustive information is often unavailable, in particular, due to the high dimensionality in the presence of prominent dependencies inherent to the considered problem.
A more realistic assumption is the knowledge of the marginal distributions and correlation matrix of $\u$.
For concreteness, we shall orientate our framework towards the practical scenario.
One should keep in mind, though, that, in general, the marginals and correlation matrix are not sufficient to recover the joint distribution.
Denote by $\{ \distribution_{\u_i} \}_{i = 1}^{\nparams \nprocs}$ and $\mCorr_\u \in \real^{\nparams \nprocs \times \nparams \nprocs}$ the marginal distribution functions and correlation matrix of $\vu$ in \eref{uncertain-parameters}, respectively.
Note that the number of distinct marginals is only $\nparams$ since $\nprocs$ components of $\vu$ correspond to the same uncertain parameter.

\subsection{Parameter Preprocessing} \slab{parameter-preprocessing}
Our foremost task is to transform $\vu$ into mutually independent \rvs\ as independence is a prerequisite of the forthcoming mathematical treatment.
To this end, an adequate probability transformation should be undertaken depending on the available information; see \cite{eldred2008} for an overview.
One transformation for which the assumed knowledge about $\vu$ is sufficient is the Nataf transformation \cite{li2008}.
Denote this transformation by
\begin{equation} \elab{probability-transformation}
  \vu = \transformation{\vz},
\end{equation}
which relates $\nparams \nprocs$ dependent \rvs, \ie, $\vu$, with $\nvars$ independent \rvs
\begin{equation} \elab{independent-random-variables}
  \vz = (\z_i)_{i = 1}^\nvars.
\end{equation}
Regardless of the marginals, $\z_i \sim \gaussianDistribution{0, 1}$, $i = 1, \dots, \nvars$, that is, each $\z_i$ has the standard Gaussian distribution.
Refer to \xref{probability-transformation} for further details about the Nataf transformation.

As we shall discuss later on, the stochastic dimensionality $\nvars$ of the problem has a considerable impact of the computational complexity of our framework.
Therefore, an important part of the preprocessing stage is model order reduction.
To this end, we preserve only those dimensions whose contribution to the total variance of $\vu$ is the most significant, which is identified by the eigenvalues of $\mCorr_\u$:
\begin{equation} \elab{dimension-contribution}
  \dimensionContribution = (\lambda_i)_{i = 1}^{\nparams \nprocs}, \hspace{1em} \norm[1]{\dimensionContribution} = 1,
\end{equation}
as it is further discussed in \xref{model-order-reduction}.
Without introducing additional transformations, we let $\transformation$ be augmented with such a reduction procedure.
Thus, we have that $\nvars \leq \nparams \nprocs$.

Let us turn to the illustrative application.
Recall that we exemplify our framework considering the effective channel length and gate-oxide thickness with the notation given in \eref{application-uncertain-parameters}.
Both parameters correspond to Euclidean distances; they take values on bounded intervals of the positive part of the real line.
With this in mind, we model the two process parameters using the four-parametric family of beta distributions:
\begin{equation*}
  \u_i \sim \distribution_{\u_i} = \betaDistribution{a_i, b_i, c_i, d_i}
\end{equation*}
where $i = 1, \dots, \nparams \nprocs$, $a_i$ and $b_i$ control the shape of the distributions, and $[ c_i, d_i ]$ correspond their supports.


\subsection{Surrogate Construction}
\begin{algorithm}
  \caption{Surrogate construction \alab{surrogate-construction}}
  \begin{algorithmic}[1]
    \vspace{0.4em}

    \Require{Algorithm X} \Comment{the subroutine evaluating $\w$}
    \Ensure{$\coefficient{\mW} \in \real^\ncorder$} \Comment{the expansion coefficients}

    \vspace{0.5em}

    \For{$i \gets 1 \textrm{ to } \nqorder$} \Comment{for each quadrature point $\point_i$}
      \Let{$\vu$}{$\transformation{\point_i}$}
      \Let{$\mW(i)$}{call {\bf Algorithm X} for $\vu$}
    \EndFor

    \Let{$\coefficient{\mW}$}{$\projection \, \mW$}

    \vspace{0.4em}
  \end{algorithmic}
\end{algorithm}
\vspace{-0.5em}


\subsection{Post-processing} \slab{post-processing}
The function given by \eref{spectral-decomposition} is nothing more than a polynomial; hence, it is easy to interpret and easy to evaluate.
Consequently, having constructed such an expansion, various statistics about $\w$ can be estimated with little effort.
Moreover, \eref{spectral-decomposition} yields analytical formulae for the expected value and variance of $\w$ solely based on the coefficients of \eref{spectral-decomposition}:
\begin{equation} \elab{probabilistic-moments}
  \expectation{\w} = \coefficient{\w}_\vZero \hspace{1em} \text{and} \hspace{1em} \variance{\w} = \sum_{\multiindex \in \multiindexSet{\level} \setminus \{ \vZero \}} \coefficient{\w}_{\multiindex}^2
\end{equation}
where $\vZero = (0)$ is a multi-index with all entries equal to zero.
Such quantities as the cumulative distribution and probability density functions can be estimated by sampling \eref{spectral-decomposition}; each sample is a trivial evaluation of a polynomial.

\begin{remark} \rlab{multiple-dimensions}
When $\w$ is multidimensional, we shall consider it as a row vector with an appropriate number of elements.
Then all the operation with respect to $\w$, such as those in \eref{spectral-decomposition}, \eref{numerical-integration}, and \eref{probabilistic-moments}, should be undertaken elementwise.
In \eref{coefficient-evaluation}, \eref{quantity-evaluation}, and \aref{surrogate-construction}, $\evw$ and $\coefficient{\evw}$ are to be treated as matrices with $\ncorder$ rows, and $\w_i$ as a row vector.
The output of Algorithm~X is assumed to be automatically reshaped into a row vector.
\end{remark}

