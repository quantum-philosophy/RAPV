In this section, we elaborate on the performance of the proposed techniques.
All the experiments are conducted on a \abbr{GNU}/Linux machine equipped with 16 processors Intel Xeon E5520 2.27~\abbr{GH}z and 24~\abbr{GB} of \abbr{RAM}.

We consider a 45-nm technological process and rely on the 45-nm standard-cell library published and maintained by NanGate \cite{nangate}.
The effective channel length and gate-oxide thickness are assumed to have nominal values equal to 22.5~nm and 1~nm, respectively.
Following the information about process variation reported by \abbr{ITRS} \cite{itrs}, we assume that each process parameter can deviate up to 12\% of its nominal value, and this percentage is treated as three standard deviations.
The corresponding probabilistic model is the one described in \sref{parameter-preprocessing} where the correlation function is taken from \cite{ukhov2014}, and the model-order reduction procedure is set to preserve 95\% of the variance of the problem (see also \xref{model-order-reduction}).
The tuning parameter $\anisotropyKnob$ in \eref{dimension-anisotropy} is set to 0.25.

Heterogeneous platforms and periodic applications are generated randomly using \abbr{TGFF} \cite{dick1998} in such a way that the execution time of tasks is uniformly distributed between 10 and 20~ms, and their dynamic power between 8 and 12~W.
The floorplans of the platforms are regular grids wherein each processing element occupies $2 \times 2\,\text{mm}^2$.
The granularity of power and temperature profiles, that is, $\dt$ in \sref{temperature-solution} and \xref{temperature-solution}, is set to 1~ms.
The stopping condition in \aref{temperature-solution} is a decrease of the normalized root-mean-square error between two successive temperature profiles smaller than 1\%, which typically requires 3--5 iterations.
In addition to the reduction of the stochastic dimensionality $\nvars$, we reduce the state-space dimensionality $\nnodes$ of the thermal system in \eref{thermal-model}.
In this case, we discard the nodes with the smallest Hankel singular values and alter the remaining ones to preserve the \abbr{DC} gain of the system.
The reduction procedure is configured to preserve 95\% of the energy of the system while ensuring that the decrease of $\nnodes$ is at most 40\%.

The leakage model needed for the calculation of $\mP_\static(\vu, \mQ)$ in \aref{temperature-solution} is based on \abbr{SPICE} simulations of a series of \abbr{CMOS} invertors taken from the NanGate cell library and configured according to the high-performance 45-nm \abbr{PTM} \cite{ptm}.
The simulations are performed for a fine-grained and sufficiently broad three-dimensional grid comprising the effective channel length, gate-oxide thickness, and temperature; the results are tabulated.
The interpolation facilities of \abbr{MATLAB} \cite{matlab} are then utilized whenever we need to evaluate the leakage power for a particular point within the range of the grid.
The output of the constructed leakage model is scaled up to account for about 40\% of the total power dissipation \cite{liu2007}.

Since the reliability optimization in \sref{reliability-optimization} embraces all the techniques developed throughout the paper, that is, in \sref{uncertainty-analysis}, \sref{temperature-analysis}, and \sref{reliability-analysis}, we shall perform our assessment directly in the context of that optimization.

\subsection{Calibration} \slab{experimental-results-calibration}
Since the proposed framework is to be placed inside an intensive design-space exploration loop (discussed in the next subsection), our foremost objective is to identify such a configuration of the framework which has a low computational demand and is still sufficiently accurate.
In this regard, the quantity of interest in \eref{quantity-of-interest} plays the key role as the objective function in \eref{objective} and the constraints in \eref{thermal-constraint} and \eref{reliability-constraint} are entirely based on it.
Therefore, we first assess a single application of the proposed framework to $\vw$ in \eref{quantity-of-interest}.
To this end, we shall compare our performance with the performance of Monte Carlo (\MC) sampling.
The operations performed by the \MC-based approach for one sample are the same as those performed by our technique for one quadrature point.
The only exception is that no reduction of any kind is undertaken inside \MC\ simulations in order to keep the corresponding results accurate.
In what follows, the number of \MC\ samples is set to $10^4$.

\begin{table}[t]
  \caption{Assessment of the accuracy}
  \vspace{-0.5em}
  \begin{tabular}{=R{10pt}-R{16pt}-R{16pt}-R{45pt}-R{45pt}-R{45pt}}
    \toprule
    $\nclevel$ & $\ncorder$ & $\nqorder$ & \errorE, \KLD & \errorQ, \KLD & \errorT, \KLD \\
    \cmidrule( r){1-3}
    \cmidrule(l ){4-6}
    1 &   3 &   5 & 0.0323 & 0.1949 & 0.4762 \\
    2 &  10 &  21 & 0.0085 & 0.0222 & 0.0390 \\
    3 &  22 &  69 & 0.0023 & 0.0033 & 0.0043 \\
    4 &  49 & 193 & 0.0019 & 0.0031 & 0.0044 \\
    5 & 111 & 589 & 0.0015 & 0.0029 & 0.0041 \\
    \bottomrule
  \end{tabular}
  \tlab{accuracy}
\end{table}
%
% Energy:
%        PC order  PC terms  QD level  QD nodes |         E, 10^4 |       Var, 10^4 |         f, 10^4 |
%     1         1         3         1         5 |          0.1458 |         12.7697 |          0.0323 |
%     2         2        10         2        21 |          0.0115 |          2.3359 |          0.0085 |
%     3         3        22         3        69 |          0.0145 |          2.9475 |          0.0023 |
%     4         4        49         4       193 |          0.0279 |          0.8484 |          0.0019 |
%     5         5       111         5       589 |          0.0226 |          0.8233 |          0.0015 |
%
% Temperature:
%        PC order  PC terms  QD level  QD nodes |         E, 10^4 |       Var, 10^4 |         f, 10^4 |
%     1         1         3         1         5 |          0.0751 |         29.0837 |          0.1949 |
%     2         2        10         2        21 |          0.0225 |          8.1120 |          0.0222 |
%     3         3        22         3        69 |          0.0256 |          3.4013 |          0.0033 |
%     4         4        49         4       193 |          0.0259 |          3.4051 |          0.0031 |
%     5         5       111         5       589 |          0.0254 |          2.7493 |          0.0029 |
%
% Lifetime:
%        PC order  PC terms  QD level  QD nodes |         E, 10^4 |       Var, 10^4 |         f, 10^4 |
%     1         1         3         1         5 |          0.0931 |         32.4772 |          0.4762 |
%     2         2        10         2        21 |          0.0384 |          9.3309 |          0.0390 |
%     3         3        22         3        69 |          0.0395 |          3.6111 |          0.0043 |
%     4         4        49         4       193 |          0.0463 |          5.0970 |          0.0044 |
%     5         5       111         5       589 |          0.0458 |          4.5505 |          0.0041 |

The results concerning accuracy are displayed in \tref{accuracy} where we study a quad-core system, \ie, $\nprocs = 4$, and vary the level of polynomial expansions $\nclevel$ from one to seven.
The errors for the three components of $\vw$ are denoted by \errorE, \errorQ, and \errorT, respectively.
The first one is the relative error of the expected value given in percentage, and the other two are the Kullback--Leibler divergence of the empirical probability distribution functions.
In general, the errors decrease as $\nclevel$ increases.
However, the trend is not monotonic with respect to \errorE.
\tref{accuracy} also contains the numbers of polynomial terms $\ncorder$ and quadrature points $\nqorder$ corresponding to each value of $\nclevel$.

We consider $\nclevel$ to three as it gives sufficiently accurate results

\tref{speed} displays the time needed to perform one characterization of $\vw$.
Naturally, no parallel computing is utilized in these experiments.
\begin{table}[t]
  \centering
  \caption{Assessment of the computational speed}
  \vspace{-0.5em}
  \begin{tabular*}{1\linewidth}{R{10pt}R{16pt}R{16pt}R{16pt}R{16pt}R{44pt}R{49pt}}
    \toprule
    $\nprocs$ & $\nnodes$ & $\nvars$ & $\ncorder$ & $\nqorder$ & Time, s & Speedup, times \\
    \cmidrule( r){1-5}
    \cmidrule(l ){6-7}
     2 & 12 &  4 & 19 &  57 & 0.19 & 175.44 \\
     4 & 16 &  6 & 22 &  69 & 0.28 & 144.93 \\
     8 & 26 &  8 & 27 &  81 & 0.92 & 123.46 \\
    16 & 45 & 10 & 30 &  93 & 1.35 & 107.53 \\
    32 & 84 & 10 & 33 & 101 & 2.42 &  99.01 \\
    \bottomrule
  \end{tabular*}
  \tlab{speed}
\end{table}


\subsection{Optimization} \slab{experimental-results-optimization}
In this subsection, we report the result of the reliability optimization discussed in \sref{reliability-optimization}.
The genetic algorithm is configured as follows.
The population contains $4 \ntasks$ individuals which are initialized using uniform distributions.
The parents for the next generation are chosen by a tournament selection with the number of competitors equal to 20\% of $\ntasks$.
A one-point crossover is then applied to 80\% of the parents.
Each parent undergoes a uniform mutation wherein each gene is altered with probability 0.01.
The top five-percent individuals always survive from one generation to the next one.
The stopping condition is the absence of improvement within 20 successive generations.

In this subsection, we report the result of the energy-driven reliability-aware optimization procedure presented in \sref{reliability-optimization}.
Since the ability to rapidly explore the design space is crucial, apart from the two reduction procedures mentioned earlier, we appeal to a number of auxiliary strategies and techniques.

The first one concerns the evaluation of a chromosome's fitness.
We begin by checking the timing constraint in \eref{timing-constraint} as it does not require any probabilistic analysis; the constraint is purely deterministic.
If \eref{timing-constraint} is violated, we set the fitness to the amount of this violation relative to the constraint---that is, to the difference between the actual application period and the deadline $\t_\maximal$ divided by $\t_\maximal$---and add a large constant, say, $C$, on top.
If \eref{timing-constraint} is satisfied, we perform our probabilistic analysis and proceed to \eref{thermal-constraint} and \eref{reliability-constraint}.
If any of the two is violated, we set the fitness to the total relative amount of violation plus $C/2$.
If all the constraints are satisfied, the fitness is set to the expected consumption of energy, as in shown in \eref{objective}.

Secondly, we make use of caching: the fitness value of each evaluated chromosome is stored in memory and pulled out when a chromosome with the same set of genes is encountered.

Finally, we rely on parallel computing: in each generation, unseen (not cached) individuals are assessed in parallel using 16 \abbr{CPU} cores, which is undertaken by virtue of the parallel computing toolbox of \abbr{MATLAB} \cite{matlab}.

The goal of the experiments in this subsection is to justify the following assertion: reliability analysis has to account for the effect of process variation on temperature.
We would like to demonstrate that the treatment of temperature as a deterministic quantity can severely compromise design decisions.
To this end, for each configuration, we shall the optimization procedure twice: once using the exact setup that has been described so far and once making the objective in \eref{objective} and the constraints in \eref{thermal-constraint} and \eref{reliability-constraint} deterministic.
To elaborate, the deterministic run assumes that that process parameters have nominal values, and, hence, it needs to perform only one system simulation to evaluate the fitness function.
In this case, \eref{objective}, \eref{thermal-constraint}, and \eref{reliability-constraint} become, respectively,
\[
  \min_{\mapping, \priority} \qoiE(\mapping, \priority), \hspace{0.7em} \qoiQ(\mapping, \priority) \geq \q_\maximal, \hspace{0.7em} \text{and} \hspace{0.7em} \qoiT(\mapping, \priority) \leq \T_\minimal.
\]

In this subsection, we report the result of the energy-driven reliability-aware optimization procedure presented in \sref{reliability-optimization}.
Since the ability to rapidly explore the design space is crucial, apart from the two reduction procedures mentioned earlier, we appeal to a number of auxiliary strategies and techniques.

The first one concerns the evaluation of a chromosome's fitness.
We begin by checking the timing constraint in \eref{timing-constraint} as it does not require any probabilistic analysis; the constraint is purely deterministic.
If \eref{timing-constraint} is violated, we set the fitness to the amount of this violation relative to the constraint---that is, to the difference between the actual application period and the deadline $\t_\maximal$ divided by $\t_\maximal$---and add a large constant, say, $C$, on top.
If \eref{timing-constraint} is satisfied, we perform our probabilistic analysis and proceed to \eref{thermal-constraint} and \eref{reliability-constraint}.
If any of the two is violated, we set the fitness to the total relative amount of violation plus $C/2$.
If all the constraints are satisfied, the fitness is set to the expected consumption of energy, as in shown in \eref{objective}.

Secondly, we make use of caching: the fitness value of each evaluated chromosome is stored in memory and pulled out when a chromosome with the same set of genes is encountered.

Finally, we rely on parallel computing: in each generation, unseen (not cached) individuals are assessed in parallel using 16 \abbr{CPU} cores, which is undertaken by virtue of the parallel computing toolbox of \abbr{MATLAB} \cite{matlab}.

The goal of the experiments in this subsection is to justify the following assertion: reliability analysis has to account for the effect of process variation on temperature.
We would like to demonstrate that the treatment of temperature as a deterministic quantity can severely compromise design decisions.
To this end, for each configuration, we shall the optimization procedure twice: once using the exact setup that has been described so far and once making the objective in \eref{objective} and the constraints in \eref{thermal-constraint} and \eref{reliability-constraint} deterministic.
To elaborate, the deterministic run assumes that that process parameters have nominal values, and, hence, it needs to perform only one system simulation to evaluate the fitness function.
In this case, \eref{objective}, \eref{thermal-constraint}, and \eref{reliability-constraint} become, respectively,
\[
  \min_{\mapping, \priority} \qoiE(\mapping, \priority), \hspace{0.7em} \qoiQ(\mapping, \priority) \geq \q_\maximal, \hspace{0.7em} \text{and} \hspace{0.7em} \qoiT(\mapping, \priority) \leq \T_\minimal.
\]

In this subsection, we report the result of the energy-driven reliability-aware optimization procedure presented in \sref{reliability-optimization}.
Since the ability to rapidly explore the design space is crucial, apart from the two reduction procedures mentioned earlier, we appeal to a number of auxiliary strategies and techniques.

The first one concerns the evaluation of a chromosome's fitness.
We begin by checking the timing constraint in \eref{timing-constraint} as it does not require any probabilistic analysis; the constraint is purely deterministic.
If \eref{timing-constraint} is violated, we set the fitness to the amount of this violation relative to the constraint---that is, to the difference between the actual application period and the deadline $\t_\maximal$ divided by $\t_\maximal$---and add a large constant, say, $C$, on top.
If \eref{timing-constraint} is satisfied, we perform our probabilistic analysis and proceed to \eref{thermal-constraint} and \eref{reliability-constraint}.
If any of the two is violated, we set the fitness to the total relative amount of violation plus $C/2$.
If all the constraints are satisfied, the fitness is set to the expected consumption of energy, as in shown in \eref{objective}.

Secondly, we make use of caching: the fitness value of each evaluated chromosome is stored in memory and pulled out when a chromosome with the same set of genes is encountered.

Finally, we rely on parallel computing: in each generation, unseen (not cached) individuals are assessed in parallel using 16 \abbr{CPU} cores, which is undertaken by virtue of the parallel computing toolbox of \abbr{MATLAB} \cite{matlab}.

The goal of the experiments in this subsection is to justify the following assertion: reliability analysis has to account for the effect of process variation on temperature.
We would like to demonstrate that the treatment of temperature as a deterministic quantity can severely compromise design decisions.
To this end, for each configuration, we shall the optimization procedure twice: once using the exact setup that has been described so far and once making the objective in \eref{objective} and the constraints in \eref{thermal-constraint} and \eref{reliability-constraint} deterministic.
To elaborate, the deterministic run assumes that that process parameters have nominal values, and, hence, it needs to perform only one system simulation to evaluate the fitness function.
In this case, \eref{objective}, \eref{thermal-constraint}, and \eref{reliability-constraint} become, respectively,
\[
  \min_{\mapping, \priority} \qoiE(\mapping, \priority), \hspace{0.7em} \qoiQ(\mapping, \priority) \geq \q_\maximal, \hspace{0.7em} \text{and} \hspace{0.7em} \qoiT(\mapping, \priority) \leq \T_\minimal.
\]

\input{include/tables/optimization.tex}
We consider platforms with 2, 4, 8, 16, and 32 cores.
Ten applications with the number of tasks equal to $20 \, \nprocs$ are randomly generated for each platform; thus, 50 applications in total.
The floorplans of the platforms and the task graphs of the applications, including the execution time and dynamic power consumption of each task for each core, are available online at \cite{sources}.
$\pr_\burn$ and $\pr_\wear$ in \eref{thermal-constraint} and \eref{reliability-constraint}, respectively, are set to 0.01.
Due to the diversity of the problems, $\t_\maximal$, $\q_\maximal$, and $\T_\minimal$ in \eref{timing-constraint}, \eref{thermal-constraint}, and \eref{reliability-constraint}, respectively, are found individually for each problem, ensuring that they make sense for the subsequent optimization.

The obtained results are reported in \tref{optimization}.
No figures regarding the reduction of the probabilistic/deterministic objective function are displayed here as they are irrelevant for the goal of this experiment established earlier.
The most important message is in the last column of \tref{optimization}.
\emph{Failure rate} refers to the ratio of the solutions discovered by the deterministic optimization that, after being reevaluated using the probabilistic approach, have been found to be faulty.
For the dual-core platform, for instance, four out of ten schedules proposed by the deterministic approach turned out to be violating the constraints on the maximal temperature and minimal lifetime with high probabilities.
The more complex the problem becomes, the higher values the failure rate attains: with 16 and 32 processing elements (320 and 640 tasks, respectively), all the deterministic solutions have been rejected.
Moreover, the constraint-violation probabilities of the deterministic solutions were found to be as high as 80\%, which by no means is acceptable.
Consequently, when analyzing reliability-related aspects of an electronic system, the ignorance of the effect of process variation on temperature can decrease the usefulness of the corresponding design decisions in the best case and turns them into dangerous, harmful decisions in the worst scenario.

We consider platforms with 2, 4, 8, 16, and 32 cores.
Ten applications with the number of tasks equal to $20 \, \nprocs$ are randomly generated for each platform; thus, 50 applications in total.
The floorplans of the platforms and the task graphs of the applications, including the execution time and dynamic power consumption of each task for each core, are available online at \cite{sources}.
$\pr_\burn$ and $\pr_\wear$ in \eref{thermal-constraint} and \eref{reliability-constraint}, respectively, are set to 0.01.
Due to the diversity of the problems, $\t_\maximal$, $\q_\maximal$, and $\T_\minimal$ in \eref{timing-constraint}, \eref{thermal-constraint}, and \eref{reliability-constraint}, respectively, are found individually for each problem, ensuring that they make sense for the subsequent optimization.

The obtained results are reported in \tref{optimization}.
No figures regarding the reduction of the probabilistic/deterministic objective function are displayed here as they are irrelevant for the goal of this experiment established earlier.
The most important message is in the last column of \tref{optimization}.
\emph{Failure rate} refers to the ratio of the solutions discovered by the deterministic optimization that, after being reevaluated using the probabilistic approach, have been found to be faulty.
For the dual-core platform, for instance, four out of ten schedules proposed by the deterministic approach turned out to be violating the constraints on the maximal temperature and minimal lifetime with high probabilities.
The more complex the problem becomes, the higher values the failure rate attains: with 16 and 32 processing elements (320 and 640 tasks, respectively), all the deterministic solutions have been rejected.
Moreover, the constraint-violation probabilities of the deterministic solutions were found to be as high as 80\%, which by no means is acceptable.
Consequently, when analyzing reliability-related aspects of an electronic system, the ignorance of the effect of process variation on temperature can decrease the usefulness of the corresponding design decisions in the best case and turns them into dangerous, harmful decisions in the worst scenario.

We consider platforms with 2, 4, 8, 16, and 32 cores.
Ten applications with the number of tasks equal to $20 \, \nprocs$ are randomly generated for each platform; thus, 50 applications in total.
The floorplans of the platforms and the task graphs of the applications, including the execution time and dynamic power consumption of each task for each core, are available online at \cite{sources}.
$\pr_\burn$ and $\pr_\wear$ in \eref{thermal-constraint} and \eref{reliability-constraint}, respectively, are set to 0.01.
Due to the diversity of the problems, $\t_\maximal$, $\q_\maximal$, and $\T_\minimal$ in \eref{timing-constraint}, \eref{thermal-constraint}, and \eref{reliability-constraint}, respectively, are found individually for each problem, ensuring that they make sense for the subsequent optimization.

The obtained results are reported in \tref{optimization}.
No figures regarding the reduction of the probabilistic/deterministic objective function are displayed here as they are irrelevant for the goal of this experiment established earlier.
The most important message is in the last column of \tref{optimization}.
\emph{Failure rate} refers to the ratio of the solutions discovered by the deterministic optimization that, after being reevaluated using the probabilistic approach, have been found to be faulty.
For the dual-core platform, for instance, four out of ten schedules proposed by the deterministic approach turned out to be violating the constraints on the maximal temperature and minimal lifetime with high probabilities.
The more complex the problem becomes, the higher values the failure rate attains: with 16 and 32 processing elements (320 and 640 tasks, respectively), all the deterministic solutions have been rejected.
Moreover, the constraint-violation probabilities of the deterministic solutions were found to be as high as 80\%, which by no means is acceptable.
Consequently, when analyzing reliability-related aspects of an electronic system, the ignorance of the effect of process variation on temperature can decrease the usefulness of the corresponding design decisions in the best case and turns them into dangerous, harmful decisions in the worst scenario.

In order to speed up the computational process, we employ a number of auxiliary techniques.
First of all, we perform model-order reduction of the thermal model in \eref{thermal-model}, which decreases the dimensionality of the state space of the thermal system given by $\nnodes$.
Second, we cache the fitness value of each evaluated chromosome such that we do not need to recompute it for other chromosomes that have the same genes.
Third, in each generation, the individuals are assessed in parallel using 12 processing cores, which is undertaken by virtue of the parallel computing toolbox of \abbr{MATLAB} \cite{matlab}.
