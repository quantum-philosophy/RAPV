In this section, we elaborate on the performance of the proposed techniques.
All the experiments are conducted on a \abbr{GNU}/Linux machine equipped with 16 processors Intel Xeon E5520 2.27~\abbr{GH}z and 24~\abbr{GB} of \abbr{RAM}.

We consider a 45-nm technological process and rely on the 45-nm standard-cell library published and maintained by NanGate \cite{nangate}.
The effective channel length and gate-oxide thickness are assumed to have nominal values equal to 22.5~nm and 1~nm, respectively.
Following the information about process variation reported by \abbr{ITRS} \cite{itrs}, we assume that each process parameter can deviate up to 12\% of its nominal value, and this percentage is treated as three standard deviations.
The corresponding probabilistic model is the one described in \sref{parameter-preprocessing} where the correlation function is taken from \cite{ukhov2014}, and the model-order reduction procedure is set to preserve 95\% of the variance of the problem (see also \xref{model-order-reduction}).

Heterogeneous platforms and periodic applications are generated randomly using \abbr{TGFF} \cite{dick1998} in such a way that the execution time of tasks is uniformly distributed between 10 and 20~ms, and their dynamic power between 8 and 12~W.
The floorplans of the platforms are regular grids wherein each processing element occupies $2 \times 2\,\text{mm}^2$.
The granularity of power and temperature profiles, that is, $\dt$ in \sref{temperature-solution} and \xref{temperature-solution}, is set to 1~ms.
The stopping condition in \aref{temperature-solution} is a decrease of the normalized root-mean-square error between two successive temperature profiles smaller than 1\%, which typically requires 3--5 iterations.
In addition to the reduction of the stochastic dimensionality $\nvars$, we reduce the state-space dimensionality $\nnodes$ of the thermal system in \eref{thermal-model}.
In this case, we discard the nodes with the smallest Hankel singular values and alter the remaining ones to preserve the \abbr{DC} gain of the system.
The reduction procedure is configured to preserve 95\% of the energy of the system while ensuring that the decrease of $\nnodes$ is at most 40\%.

The leakage model needed for the calculation of $\mP_\static(\vu, \mQ)$ in \aref{temperature-solution} is based on \abbr{SPICE} simulations of a series of \abbr{CMOS} invertors taken from the NanGate cell library and configured according to the high-performance 45-nm \abbr{PTM} \cite{ptm}.
The simulations are performed for a fine-grained and sufficiently broad three-dimensional grid comprising the effective channel length, gate-oxide thickness, and temperature; the results are tabulated.
The interpolation facilities of \abbr{MATLAB} \cite{matlab} are then utilized whenever we need to evaluate the leakage power for a particular point within the range of the grid.
The output of the constructed leakage model is scaled up to account for about 40\% of the total power dissipation \cite{liu2007}.

Since the reliability optimization in \sref{reliability-optimization} embraces all the techniques developed throughout the paper, that is, in \sref{uncertainty-analysis}, \sref{temperature-analysis}, and \sref{reliability-analysis}, we shall perform our assessment directly in the context of that optimization.

\subsection{Calibration} \slab{experimental-results-calibration}
As described in \sref{reliability-optimization}, the driving force of our optimization is a genetic algorithm; in this regard, the quantity of interest in \eref{quantity-of-interest} plays the key role since the objective function in \eref{objective} and the constraints in \eref{thermal-constraint} and \eref{reliability-constraint} are entirely based on it.
Hence, we begin with an assessment of the accuracy and computational speed of the proposed solutions applied to $\vw$ in \eref{quantity-of-interest}, which also servers a calibration stage for the forthcoming design-space exploration.
To this end, we shall compare our performance with the performance of Monte Carlo (\MC) sampling.
The operations performed by the \MC-based approach for one sample are the same as those performed by our technique for one quadrature point.
The only exception is that no reduction of any kind is undertaken inside \MC\ simulations in order to keep the corresponding results accurate.
In the following experiments, the number of \MC\ samples is set to $10^4$.

\begin{table*}
  \caption{Assessment of the accuracy}
  \vspace{-0.5em}
  \begin{tabular}{=R{14pt}-R{16pt}-R{17pt}-R{36pt}-R{36pt}-R{36pt}-R{36pt}-R{36pt}-R{36pt}-R{36pt}-R{36pt}-R{36pt}}
    \toprule
    & & & \multicolumn{3}{c}{Energy consumption} & \multicolumn{3}{c}{Maximal temperature} & \multicolumn{3}{c}{Reliability parametrization} \\
    $\nclevel$ & $\ncorder$ & $\nqorder$ & \eE, \% & \eV, \% & \eD, \KLD & \eE, \% & \eV, \% & \eD, \KLD & \eE, \% & \eV, \% & \eD, \KLD \\
    \cmidrule( r){1-3}
    \cmidrule(lr){4-6}
    \cmidrule(lr){7-9}
    \cmidrule(l ){10-12}
    1 &   3 &    5 & 0.1421 & 13.2754 & 0.0356 & 0.0725 & 29.4988 & 0.2110 & 0.0910 & 33.0563 & 0.5285 \\
    2 &  10 &   21 & 0.0229 &  2.0183 & 0.0083 & 0.0190 &  7.5465 & 0.0191 & 0.0256 &  8.4358 & 0.0345 \\
    3 &  22 &   69 & 0.0245 &  2.8427 & 0.0024 & 0.0201 &  3.0347 & 0.0021 & 0.0259 &  2.9822 & 0.0034 \\
    4 &  49 &  193 & 0.0304 &  1.3074 & 0.0019 & 0.0236 &  3.6327 & 0.0030 & 0.0394 &  5.2388 & 0.0040 \\
    5 & 111 &  589 & 0.0220 &  1.1144 & 0.0015 & 0.0207 &  1.6341 & 0.0023 & 0.0285 &  2.7353 & 0.0036 \\
    \bottomrule
  \end{tabular}
  \tlab{accuracy}
  \vspace{-1.0em}
\end{table*}

The results concerning accuracy are displayed in \tref{accuracy} where we study a quad-core system, \ie, $\nprocs = 4$, and vary the level of polynomial expansions $\nclevel$ from one to seven.
For each of the three components of $\vw$, we report three error metrics: the relative errors of the expected value and variance and the Kullback--Leibler divergence of the empirical probability distribution functions.
The three metrics are denoted by \eE, \eV, and \eD, respectively.
The numbers of polynomial terms $\ncorder$ and quadrature points $\nqorder$ corresponding to each value of $\nclevel$ can also be found in the table.
In general, the errors decrease as $\nclevel$ increases.
However, the trend is not strictly monotonic.

\tref{speed} displays the time needed to perform one characterization of $\vw$.
Naturally, no parallel computing is utilized in these experiments.
\begin{table}[t]
  \centering
  \caption{Assessment of the computational speed}
  \vspace{-0.5em}
  \begin{tabular*}{1\linewidth}{R{10pt}R{16pt}R{16pt}R{16pt}R{16pt}R{44pt}R{49pt}}
    \toprule
    $\nprocs$ & $\nnodes$ & $\nvars$ & $\ncorder$ & $\nqorder$ & Time, s & Speedup, times \\
    \cmidrule( r){1-5}
    \cmidrule(l ){6-7}
     2 & 12 &  4 & 19 &  57 & 0.29 & 175.44 \\
     4 & 16 &  6 & 22 &  69 & 0.42 & 144.93 \\
     8 & 26 &  8 & 27 &  81 & 0.82 & 123.46 \\
    16 & 45 & 10 & 30 &  93 & 2.01 & 107.53 \\
    32 & 84 & 10 & 33 & 101 & 4.12 &  99.01 \\
    \bottomrule
  \end{tabular*}
  \tlab{speed}
\end{table}


\subsection{Optimization} \slab{experimental-results-optimization}
In this subsection, we report the result of the reliability optimization discussed in \sref{reliability-optimization}.
The genetic algorithm is configured as follows.
The population contains $4 \ntasks$ individuals which are initialized using uniform distributions.
The parents for the next generation are chosen by a tournament selection with the number of competitors equal to 20\% of $\ntasks$.
A one-point crossover is then applied to 80\% of the parents.
Each parent undergoes a uniform mutation wherein each gene is altered with probability 0.01.
The top five-percent individuals always survive from one generation to the next one.
The stopping condition is the absence of improvement within 20 successive generations.

In order to speed up the computational process, we employ a number of auxiliary techniques.
First of all, we perform model-order reduction of the thermal model in \eref{thermal-model}, which decreases the dimensionality of the state space of the thermal system given by $\nnodes$.
Second, we cache the fitness value of each evaluated chromosome such that we do not need to recompute it for other chromosomes that have the same genes.
Third, in each generation, the individuals are assessed in parallel using 12 processing cores, which is undertaken by virtue of the parallel computing toolbox of \abbr{MATLAB} \cite{matlab}.
