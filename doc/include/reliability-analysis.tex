Electronic systems are naturally exposed to a wide range of failure mechanisms.
The most dominant examples include electromigration, time-dependent dielectric breakdown, stress migration, and thermal cycling \cite{xiang2010}; see \cite{jedec2011} for an exhaustive overview.
The lifetime with respect to each mechanism is typically modeled using log-normal and Weibull distributions, including mixtures of the two families.
Depending on the type of fatigue, each such model has a number of parameters.
In particular, all of the aforementioned mechanisms have exponential dependencies on the operating temperature.

Let $\T: \outcomes \to \real$ be a \rv\ representing the lifetime of the considered system.
The lifetime is the time until the system experiences a fault after which the system no longer meets its requirements.
$\T$ is a function of the individual lifetimes of the processing elements, which are denoted by a random vector $\vT = (\T_i): \outcomes \to \real^\nprocs$.
Let $\distribution_\T(\cdot | \vwr)$ be the distribution of $\vT$ where $\vwr = (\wr_i)$ is a vector of parameters.
The survival function of the system is defined as
\[
  \survival_\T(\t | \vwr) = 1 - \distribution_\T(\t | \vwr).
\]
Similarly, we shall denote by $\{ \distribution_{\T_i}(\cdot | \vwr) \}_{i = 1}^\nprocs$ and $\{ \survival_{\T_i}(\cdot | \vwr) \}_{i = 1}^\nprocs$ the marginal distributions and individual survival functions of the processing elements, respectively.

In the context of reliability analysis, our main objective is to take the state-of-the-art reliability models to the next level by relaxing the assumptions on their parametrization $\vwr$ and enriching this parametrization via probabilistic temperature analysis considering process variation.
To elaborate, our idea is based on the observation that the major part of the time associated with reliability analysis is ascribed to the evaluation of the parameters collected in $\vwr$ whereas a typical reliability model \perse, that is, when $\vwr$ is known, has a negligible cost.
Therefore, we propose to use the spectral decompositions developed in \sref{uncertainty-analysis} and \sref{temperature-analysis} in order to construct surrogate representations of $\vwr$.
In contrast to the straightforward use of MC sampling, such light representations make the subsequent analysis highly efficient from the computational perspective.

The structure of $\T$ with respect to $\{ \T_i \}_{i = 1}^\nprocs$ is problem specific, and it can be especially diverse in the context of fault-tolerant systems.
Likewise, the corresponding probabilistic models of failures depend on the fatigues that each particular system is suffering from.
Our approach provides a great flexibility in this regard since it does not make any assumptions on the utilized reliability model; it is only concerned with the corresponding parametrization $\vwr$.
Hence, the proposed solution is readily applicable to any model that the user considers to be the most adequate for the problem at hand.

Despite its generality, the proposed technique can be better appreciated considering an example.
To this end, we turn to our application and shall focus on the thermal-cycling fatigue as it has arguably the most prominent dependency on temperature: apart from average/maximal temperatures, the frequencies and amplitudes of temperature fluctuations matter in this case.
We also assume that any fault of any processing element makes the whole system fail, and $\{ \T_i \}_{i = 1}^\nprocs$ are conditionally independent given $\vwr$.
Consequently, in this scenario,
\begin{equation} \elab{reliability-assumptions}
  \T = \min_{i = 1}^\nprocs \T_i \hspace{1em} \text{and} \hspace{1em} \survival_\T(\t | \vwr) = \prod_{i = 1}^\nprocs \survival_{\T_i}(\t | \vwr).
\end{equation}
