The probability space that we shall reside in is defined as a triple $(\outcomes, \sigmaAlgebra, \probabilityMeasure)$ where $\outcomes$ is a set of outcomes, $\sigmaAlgebra \subseteq 2^\outcomes$ is a $\sigma$-algebra on $\outcomes$, and $\probabilityMeasure: \sigmaAlgebra \to [0, 1]$ is a probability measure \cite{maitre2010}.
Loosely speaking, an $n$-dimensional random variable (a random vector) is then a function $\v{X}: \o \in \outcomes \mapsto \v{X}(\o) \in \real^n$.
In what follows, the probability space $(\outcomes, \sigmaAlgebra, \probabilityMeasure)$ is always implied.

The system depends on a set of uncertain parameters, denoted by a random vector $\vU(\o)$, $\o \in \outcomes$, which manifest themselves in deviations of the actual power dissipation from nominal values and, consequently, in deviations of temperature from the one corresponding to the nominal power consumption.

The core of your framework operates on mutually independent random variables.
In general, however, the uncertain parameters that comprise $\vU(\o)$ are correlated.
Hence, an adequate probability transformation should be undertaken \cite{eldred2008}.
Denote such a transformation by $\vU(\o) = \oTransform{\vZ(\o)}$, which relates $\vU(\o)$ with $\nvars$ independent random variables $\vZ(\o)$.
