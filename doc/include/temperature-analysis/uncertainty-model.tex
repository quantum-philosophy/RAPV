The probability space that we shall reside in is defined as a triple $(\outcomes, \sigmaAlgebra, \probabilityMeasure)$ where $\outcomes$ is a set of outcomes, $\sigmaAlgebra \subseteq 2^\outcomes$ is a $\sigma$-algebra on $\outcomes$, and $\probabilityMeasure: \sigmaAlgebra \to [0, 1]$ is a probability measure \cite{maitre2010}.
Loosely speaking, an $n$-dimensional random variable (a random vector) is then a function $\v{x}: \o \in \outcomes \mapsto \v{x}(\o) \in \real^n$.
In what follows, the probability space $(\outcomes, \sigmaAlgebra, \probabilityMeasure)$ is always implied.

The system depends on a set of uncertain parameters, denoted by a random vector $\vU(\o)$, $\o \in \outcomes$, which manifest themselves in deviations of the actual power dissipation from nominal values and, consequently, in deviations of temperature from the one corresponding to the nominal power consumption.
