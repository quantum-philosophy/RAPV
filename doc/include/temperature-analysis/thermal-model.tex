Based on the information gathered in $\specification$ (see \sref{problem-formulation}), an equivalent thermal RC circuit of the system is constructed \cite{skadron2004}.
The circuit comprises $\nnodes$ thermal nodes, and its structure depends on the intended level of granularity that impacts the resulting accuracy.
For clarity, we assume that each processing element is mapped onto one corresponding node, and the thermal package is represented as a set of additional nodes.

The thermal dynamics of the system are modeled using the following system of differential-algebraic equations \cite{ukhov2012, ukhov2014}:
\begin{subnumcases}{\elab{thermal-model}}
  \frac{\d \, \vs(\t, \vu)}{\d\t} = \mA \: \vs(\t, \vu) + \mB \: \vp(\t, \vu, \vq(\t, \vu)) \elab{thermal-model-inner} \\
  \vq(\t, \vu) = \mB^T \vs(\t, \vu) + \vq_\ambient \elab{thermal-model-outer}
\end{subnumcases}
where
\[
  \mA = -\mCth^{-\frac{1}{2}} \mGth \mCth^{-\frac{1}{2}}, \hspace{1em} \text{and} \hspace{1em} \mB = \mCth^{-\frac{1}{2}} \mM.
\]
For time $\t \geq 0$, $\vp \in \real^\nprocs$, $\vq \in \real^\nprocs$, and $\vs \in \real^\nnodes$ are the power, temperature, state vectors, respectively.
$\vq_\ambient \in \real^\nprocs$ is a vector of the ambient temperature.
$\mM \in \real^{\nnodes \times \nprocs}$ is a matrix that distributes the power dissipation of the processing elements across the thermal nodes; without loss of generality, $\mM$ is a rectangular diagonal matrix wherein each diagonal element is equal to unity.
$\mCth \in \real^{\nnodes \times \nnodes}$ and $\mGth \in \real^{\nnodes \times \nnodes}$ are a diagonal matrix of the thermal capacitance and a symmetric, positive-definite matrix of the thermal conductance, respectively.
