Given the thermal specification $\specification$ of the system at hand (see \sref{problem-formulation}), an equivalent thermal RC circuit with $\nnodes$ thermal nodes is constructed \cite{skadron2004}.
The structure of the circuit depends on the intended level of granularity and, thus, impacts the resulting accuracy.
For clarity, we assume that each processing element is mapped onto one corresponding node, and the thermal package is represented as a set of additional nodes.

The thermal behavior of the constructed circuit is modeled with the following state-space system \cite{ukhov2012}:
\begin{subnumcases}{\elabel{temperature-model}}
  \frac{\d\vX(\t, \o)}{\d\t} = \mA \: \vX(\t, \o) + \mB \: \vP(\t, \o) \elabel{temperature-model-inner} \\
  \vQ(\t, \o) = \mB^T \vX(\t, \o) + \vQ_\ambient \elabel{temperature-model-outer}
\end{subnumcases}
where
\[
  \mA = -\mCth^{-\frac{1}{2}} \mGth \mCth^{-\frac{1}{2}} \hspace{1em} \text{and} \hspace{1em} \mB = \mCth^{-\frac{1}{2}} \mM.
\]
In the above equations, $\vP \in \real^\nprocs$, $\vQ \in \real^\nprocs$, and $\vX \in \real^\nnodes$ are the input power, output temperature, internal state vectors, respectively, $\mM \in \real^{\nnodes \times \nprocs}$ is a matrix mapping the processing elements onto the thermal nodes, $\mCth \in \real^{\nnodes \times \nnodes}$ and $\mGth \in \real^{\nnodes \times \nnodes}$ are a diagonal matrix of the thermal capacitance and a symmetric, positive-definite matrix of the thermal conductance, respectively, and $\vQ_\ambient \in \real^\nprocs$ is a vector of the ambient temperature.
In general, \eref{temperature-model-inner} does not have a closed-form solution due to the power term, which is an arbitrary function.

Let $\dt$ be the sampling interval of the power and temperature profiles, and $\nsteps$ be the total number of such $\dt$-spaced samples. As discussed in \cite{ukhov2012}, the solution of \eref{temperature-model} can then be reduced to the following recurrence for $k = 1, \dots, \nsteps$:
\begin{equation} \elabel{recurrence}
  \vX_{k + 1} = \mE \: \vX_k + \mF \: \vP_k
\end{equation}
where the subscript $k$ stands for the time moment $k \dt$,
\[
  \mE = e^{\mA \dt} \hspace{1em} \text{and} \hspace{1em} \mF = \mA^{-1} (e^{\mA \dt} - \mI) \: \mB.
\]
For computational efficiency, the state matrix $\mA$ is factorized using the eigenvalue decomposition as follows \cite{press2007}:
\begin{equation} \elabel{eigenvalue-decomposition}
  \mA = \mV \mL \mV^T
\end{equation}
where $\mV$ and $\mL = \diag{\lambda_i}$ are an orthogonal matrix of the eigenvectors and a diagonal matrix of the eigenvectors of $\mA$, respectively.
The matrices $\mE$ and $\mF$ are now trivial to calculate:
\begin{align*}
  & \mE = \mV \; \diag{e^{\lambda_i \dt}} \mV^T \text{ and} \\
  & \mF = \mV \; \diag{ \frac{e^{\lambda_i \dt} - 1}{\lambda_i} } \mV^T \mB.
\end{align*}
