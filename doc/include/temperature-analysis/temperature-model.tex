Based on the information gathered in $\specification$ (see \sref{problem-formulation}), an equivalent thermal RC circuit of the system is constructed \cite{skadron2004}.
The circuit comprises $\nnodes$ thermal nodes, and its structure depends on the intended level of granularity that impacts the resulting accuracy.
For clarity, we assume that each processing element is mapped onto one corresponding node, and the thermal package is represented as a set of additional nodes.

The thermal dynamics of the system are modeled using the following system of differential-algebraic equations \cite{ukhov2012}:
\begin{subnumcases}{\elabel{temperature-model}}
  \frac{\d\vX(\t, \o)}{\d\t} = \mA \: \vX(\t, \o) + \mB \: \vP(\t, \o) \elabel{temperature-model-inner} \\
  \vQ(\t, \o) = \mB^T \vX(\t, \o) + \vQ_\ambient \elabel{temperature-model-outer}
\end{subnumcases}
where
\[
  \mA = -\mCth^{-\frac{1}{2}} \mGth \mCth^{-\frac{1}{2}} \hspace{1em} \text{and} \hspace{1em} \mB = \mCth^{-\frac{1}{2}} \mM.
\]
For each time $\t$, $\vP \in \real^\nprocs$, $\vQ \in \real^\nprocs$, and $\vX \in \real^\nnodes$ are the power, temperature, state vectors, respectively.
$\vQ_\ambient \in \real^\nprocs$ is a vector of the ambient temperature.
$\mM \in \real^{\nnodes \times \nprocs}$ is a matrix that distributes the power dissipation of the processing elements across the thermal nodes; without loss of generality, $\mM$ is a rectangular diagonal matrix wherein each diagonal element is equal to unity.
$\mCth \in \real^{\nnodes \times \nnodes}$ and $\mGth \in \real^{\nnodes \times \nnodes}$ are a diagonal matrix of the thermal capacitance and a symmetric, positive-definite matrix of the thermal conductance, respectively.
\begin{remark}
In general, \eref{temperature-model-inner} is a system of ordinary differential equations which is non-linear due to the power term defined in \eref{power-model}.
Consequently, the thermal system in \eref{temperature-model} does not have a straightforward closed-form solution.
\end{remark}

In order to efficiently solve \eref{temperature-model-inner}, we utilize a numerical method from the family of so-called exponential integrators \cite{hochbruck2010} as it is described in \xref{temperature-model}.
Let $\dt$ be the sampling interval of the power and temperature profiles, and $\nsteps$ be the total number of such $\dt$-spaced samples. As discussed in \cite{ukhov2012}, the solution of \eref{temperature-model} can then be reduced to the following recurrence for $k = 1, \dots, \nsteps$:
\begin{equation} \elabel{recurrence}
  \vX_{k + 1} = \mE \: \vX_k + \mF \: \vP_k
\end{equation}
where the subscript $k$ stands for the time moment $k \dt$,
\[
  \mE = e^{\mA \dt} \hspace{1em} \text{and} \hspace{1em} \mF = \mA^{-1} (e^{\mA \dt} - \mI) \: \mB.
\]
For computational efficiency, the state matrix $\mA$ is factorized using the eigenvalue decomposition as follows \cite{press2007}:
\begin{equation} \elabel{eigenvalue-decomposition}
  \mA = \mV \mL \mV^T
\end{equation}
where $\mV$ and $\mL = \diag{\lambda_i}$ are an orthogonal matrix of the eigenvectors and a diagonal matrix of the eigenvectors of $\mA$, respectively.
The matrices $\mE$ and $\mF$ are now trivial to calculate:
\begin{align*}
  & \mE = \mV \; \diag{e^{\lambda_i \dt}} \mV^T \text{ and} \\
  & \mF = \mV \; \diag{ \frac{e^{\lambda_i \dt} - 1}{\lambda_i} } \mV^T \mB.
\end{align*}
