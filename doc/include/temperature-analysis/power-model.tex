Recall that the system is composed of $\nprocs$ processing elements and has a dependency on the outcome of the probability space via $\vu(\o)$, $\o \in \outcomes$.
The total dissipation of power is modeled as the following system of $\nprocs$ temporal stochastic process:
\begin{equation} \elab{power-model}
  \vP(\t, \vu) = \vP_\dynamic(\t) + \vP_\static(\vQ(\t, \vu), \vu)
\end{equation}
where, for each time $\t \geq 0$, $\vP_\dynamic \in \real^\nprocs$ and $\vP_\static \in \real^\nprocs$ are random vectors representing the dynamic and static components of the total power, respectively, and $\vQ \in \real^\nprocs$ is the corresponding random vector of temperature.
\begin{remark}
In \eref{power-model}, \textnormal{$\vP_\dynamic$} has no dependency on $\vu(\o)$ as the influence of process variation on the dynamic power is known to be negligibly small \cite{srivastava2010}.
On the other hand, the variability of \textnormal{$\vP_\static$} is substantial and is further magnified by the well-known interdependency between leakage and temperature.
\end{remark}
