Due to the inherent complexity, uncertainty quantification problems are typically viewed as approximation problems: one constructs a computationally efficient surrogate of the stochastic model under consideration and then studies this light representation instead.
The theory of polynomial chaos (PC) expansions \cite{maitre2010} is one way to construct such an approximation, in which the approximating functions are polynomials orthogonal with respect to certain probability measures.

\subsubsection{Preprocessing of the uncertain parameters}
The construction of a PC expansion is based on mutually independent random variables.
In general, however, the uncertain parameters that comprise $\vU(\o)$ are correlated.
Hence, an adequate probability transformation should be undertaken \cite{eldred2008}.
Denote such a transformation by $\vU(\o) = \oTransform{\vZ(\o)}$, which relates $\vU(\o)$ with $\nvars$ independent random variables $\vZ(\o)$.

\subsubsection{Choice of a polynomial basis}
The next step towards a PC expansion is the choice of a suitable polynomial basis $\{ \pcb_i(\vZ) \}_{i = 1}^\infty$.
Each $\pcb_i(\vZ)$ is a real-valued polynomial in terms of $\nvars$ variables (recall that $\vZ(\o) \in \real^\nvars$).
This choice depends mainly on the probability distributions of $\vZ(\o)$.
Many discrete and continuous distributions directly correspond to certain families of orthogonal polynomials found in the so-called Askey scheme of hypergeometric orthogonal polynomials; see, \eg, \cite{eldred2008}.
For instance, the Hermite polynomials are a natural choice for Gaussian distributions.

\subsubsection{Expansion of the quantity of interest}
The primary uncertain quantity of this paper is the DSS temperature profile, denoted by the random matrix $\mQ(\o) \in \real^{\nprocs \times \nsteps}$,\footnote{As always, the argument $\o$ stands for $\vU(\o)$ or $\oTransform{\vZ(\o)}$.} of the system at hand under a certain workload, represented by the matrix $\mP_\dynamic \in \real^{\nprocs \times \nsteps}$.
However, due to the reason that will become clear later on, let us denote the quantity of interest by an abstract random matrix $\mR(\o)$.
The expansion of $\mR(\o)$ is
\begin{equation} \elabel{polynomial-chaos-expansion}
  \mR(\o) = \sum_{i = 1}^\infty \pcc{\mR}_i \, \pcb_i(\vZ(\o))
\end{equation}
where $\pcc{\mR}_i$ are the coefficients of the expansion, which are matrix valued and have the same dimensionality as $\mR(\o)$.
For practical computations, \eref{polynomial-chaos-expansion} is truncated to preserve only $\nterms$ first polynomial terms.
The result is nothing more than a polynomial; hence, it is easy to interpret and easy to evaluate.
Consequently, such quantities as \cdfs\ and \pdfs\ can be estimated at no effort.
Moreover, the coefficients of a PC expansion immediately yield analytical formulae for the expected value and variance of the expanded quantity.

\subsubsection{Computation of the expansion coefficients}
The question now to discuss is the computation of $\{ \pcc{\mR}_i \}_{i = 1}^\nterms$ in \eref{polynomial-chaos-expansion}.
Each $\pcc{\mR}_i$ is the following multidimensional integral:
\begin{align}
  \pcc{\mR}_i & = \oExpectation{\mR(\oTransform{\vZ(\o)}) \, \pcb_i(\vZ(\o))} \nonumber \\
  & = \int \mR(\oTransform{\vZ}) \, \pcb_i(\vZ) \, \fPDF(\vZ) \, \d\vZ \elabel{polynomial-chaos-coefficients}
\end{align}
where $\fPDF(\vZ)$ denotes the \pdf\ of $\vZ(\o)$.\footnote{In case of a discrete distribution, the coefficients are summations with respect to the corresponding probability mass function.}
Recall that $\pcc{\mR}_i$ is a matrix and note that the operations in \eref{polynomial-chaos-coefficients} are elementwise.

The integral in \eref{polynomial-chaos-coefficients} should be taken numerically.
In numerical integration, an integral of a function is approximated by a summation over the function values computed at a set of prescribed points and multiplied by the corresponding set of prescribed weights.
Such pairs of points and weights are called quadrature rules \cite{press2007}.
When $\mR(\o)$ is $\mQ(\o)$, for any $\vZ$ in the support of the probability distribution of $\vZ(\o)$, the needed $\mQ(\oTransform{\vZ})$ can be evaluated using \aref{deterministic-solution-with-leakage} with $\vU = \oTransform{\vZ}$.
