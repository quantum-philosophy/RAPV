In this subsection, we address the case when $\o$ is fixed in \eref{power-model} (and, hence, in \eref{temperature-model}).
In other words, the parametrization $\vU(\o)$ is known, which we emphasize by $\vU$.
Assume for now that $\vP_\static$ in \eref{power-model} does not depend on $\vQ$, \ie, there is no interdependency between leakage and temperature.
Under the dynamic steady-state condition, $\vX_1 = \vX_{\nsteps + 1}$.
This boundary constraint and the recurrence in \eref{recurrence} yield a large system of $\nnodes \nsteps$ linear equations with $\nnodes \nsteps$ unknowns.
As shown in \cite{ukhov2012}, this problem can be efficiently solved by exploiting the particular structure of the system and the decomposition in \eref{eigenvalue-decomposition}.
The pseudocode of this algorithm is given in \aref{deterministic-solution}.\footnote{Hereafter, we utilize the MATLAB notation to refer to a particular column, \eg, $\m{A}(:, k)$, or a particular row, \eg, $\m{A}(k, :)$, of a matrix $\m{A}$. Also, we shall underline auxiliary vectors, \eg, $\tmp{\v{a}}$, and matrices, \eg, $\tmp{\m{A}}$.}
\begin{algorithm}
  \caption{The deterministic \dssta\ with no static dissipation of power. \alabel{deterministic-solution}}
  \begin{algorithmic}[1]
    \vspace{0.4em}

    \Require{$\mP \in \real^{\nprocs \times \nsteps}$}
    \Ensure{$\mQ \in \real^{\nprocs \times \nsteps}$}

    \vspace{0.5em}

    \Let{$\tmp{\m{A}}$}{$\mF \; \mP$}
    \Let{$\tmp{\v{a}}$}{$\tmp{\m{A}}(:, 1)$}

    \For{$k \gets 2 \textrm{ to } \nsteps$}
      \Let{$\tmp{\v{a}}$}{$\mE \; \tmp{\v{a}} + \tmp{\m{A}}(:, k)$}
    \EndFor

    \Let{$\mX(:, 1)$}{$\mV \; \diag{\left( 1 - e^{\nsteps \dt \lambda_i} \right)^{-1}} \; \mV^T \; \tmp{\v{a}}$}

    \For{$k \gets 2 \textrm{ to } \nsteps$}
      \Let{$\m{X}(:, k)$}{$\mE \; \mX(:, k - 1) + \tmp{\m{A}}(:, k - 1)$}
    \EndFor

    \Let{$\mQ$}{$\mB^T \mX + \mQ_\ambient$}

    \vspace{0.4em}
  \end{algorithmic}
\end{algorithm}


The time complexity of direct solutions of the system is $O(\nsteps^3 \nnodes^3)$ while the one of \aref{deterministic-solution} is only $O(\nsteps \nnodes^2 + \nnodes^3)$.\footnote{The complexity reported in \cite{ukhov2012} is erroneously overestimated to $O(\nsteps \nnodes^3)$.}

In order to account for the leakage-temperature interdependence, \aref{deterministic-solution} is to be repeated for a sequence of total power profiles until the corresponding sequence of temperature profiles converges or some other stopping condition is met (\eg, a maximal temperature constraint is violated).
Each power profile is computed according to \eref{power-model} with respect to the temperature profile from the previous iteration starting from the ambient temperature.
This procedure is illustrated in \aref{deterministic-solution-with-leakage}.
\begin{algorithm}
  \caption{Deterministic DSSTA with leakage. \alabel{deterministic-solution-with-leakage}}
  \begin{algorithmic}[1]
    \vspace{0.4em}

    \Require{A dynamic power profile $\mP_\dynamic$, parameters $\vU$.}
    \Ensure{The corresponding temperature profile $\mQ$.}

    \vspace{0.5em}

    \Let{$\mQ$}{$\mQ_\ambient$}

    \Repeat
      \Let{$\mP$}{$\mP_\dynamic + \mP_\static(\mQ, \vU)$}
      \Let{$\mQ^\prime$}{$\mQ$}
      \Let{$\mQ$}{run {\bf Algorithm 1} for $\mP$}
    \Until{stopping condition is met according to $\mQ$ and $\mQ^\prime$}

    \vspace{0.4em}
  \end{algorithmic}
\end{algorithm}


In \aref{deterministic-solution-with-leakage}, $\mP_\static(\mQ, \vU)$ should be understood as a call to a subroutine that returns an $\nprocs \times \nsteps$ matrix wherein the $(i, j)$th element is the static component of the power dissipation of the $i$th processing element at the $j$th moment of time with respect to the temperature given by the $(i, j)$th element of $\mQ$ and $\vU$.
