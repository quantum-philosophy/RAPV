Let us fix $\o \in \outcomes$, meaning that $\vu$ is assumed to be known, and consider the above system as deterministic.
In general, \eref{thermal-model-inner} is a system of ordinary differential equations which is nonlinear due to the power term, given in \eref{power-model} as an arbitrary function.
Hence, the system in \eref{thermal-model} does not have a general closed-form solution.
A robust and computationally efficient solution of \eref{thermal-model} for a given $\vu$ is an essential part of our probabilistic framework.
In order to attain such a solution, we utilize a numerical method from the family of so-called exponential integrators \cite{hochbruck2010}.
The procedure is described in \xref{temperature-solution}, and here we use the final result; see also \cite{ukhov2012}.

Recall that we are to analyze a dynamic power profile $\mP_\dynamic$ covering a time interval $[0, \period]$ with $\nsteps$ samples that are evenly spaced in time.
The transient solution of \eref{thermal-model-inner} is reduced to the following recurrence for $k = 1, \dots, \nsteps$:
\begin{equation} \elab{recurrence}
  \vs_k = \mE \: \vs_{k - 1} + \mF \: \vp_k
\end{equation}
where the subscript $k$ stands for time $k \dt$, $\vs_0 = \vZero$,
\[
  \mE = e^{\mA \dt}, \hspace{1em} \text{and} \hspace{1em} \mF = \mA^{-1} (e^{\mA \dt} - \mOne) \: \mB.
\]
For computational efficiency, the state matrix $\mA$ is factorized using the eigenvalue decomposition as follows:
\begin{equation} \elab{eigenvalue-decomposition}
  \mA = \mV \mL \mV^T
\end{equation}
where $\mV$ and $\mL = \diagonal{\lambda_i}$ are an orthogonal matrix of the eigenvectors and a diagonal matrix of the eigenvectors of $\mA$, respectively.
The matrices $\mE$ and $\mF$ are then
\begin{align*}
  & \mE = \mV \; \diagonal{e^{\lambda_i \dt}} \mV^T \text{ and} \\
  & \mF = \mV \; \diagonal{ \frac{e^{\lambda_i \dt} - 1}{\lambda_i} } \mV^T \mB.
\end{align*}
To sum up, the derivation up to this point is sufficient to for the \tta\ via \eref{recurrence} followed by \eref{thermal-model-outer}.
\begin{remark}
Although we always refer to temperature, each temperature analysis developed in this paper is accompanied by the corresponding power analysis as the two are inseparable due to the leakage-temperature interplay.
Consequently, when it is appropriate, one can easily extract only the (temperature-aware) power part of the presented solutions.
\end{remark}

Let us move on to the \dss\ case.
Assume for now that $\vp_\static$ in \eref{power-model} does not depend on $\vq$, \ie, there is no interdependency between leakage and temperature.
The \dss\ boundary condition is
\[
  \vs_1 = \vs_{\nsteps + 1}.
\]
This constraint and \eref{recurrence} yield a block-circulant system of $\nnodes \nsteps$ linear equations with $\nnodes \nsteps$ unknowns.
As shown in \cite{ukhov2012}, this problem can be efficiently solved by exploiting the particular structure of the system and the decomposition in \eref{eigenvalue-decomposition}.
The pseudo code of this algorithm, which delivers the exact solution under the made assumptions, is given in \aref{deterministic-solution}.
In this subsection, we address the case when $\o$ is fixed in \eref{power-model} (and, hence, in \eref{temperature-model}).
In other words, the parametrization $\vU(\o)$ is known, which we emphasize by $\vU$.
Assume for now that $\vP_\static$ in \eref{power-model} does not depend on $\vQ$, \ie, there is no interdependency between leakage and temperature.
Under the dynamic steady-state condition, $\vX_1 = \vX_{\nsteps + 1}$.
This boundary constraint and the recurrence in \eref{recurrence} yield a large system of $\nnodes \nsteps$ linear equations with $\nnodes \nsteps$ unknowns.
As shown in \cite{ukhov2012}, this problem can be efficiently solved by exploiting the particular structure of the system and the decomposition in \eref{eigenvalue-decomposition}.
The pseudocode of this algorithm is given in \aref{deterministic-solution}.\footnote{Hereafter, we utilize the MATLAB notation to refer to a particular column, \eg, $\m{A}(:, k)$, or a particular row, \eg, $\m{A}(k, :)$, of a matrix $\m{A}$. Also, we shall underline auxiliary vectors, \eg, $\tmp{\v{a}}$, and matrices, \eg, $\tmp{\m{A}}$.}
In this subsection, we address the case when $\o$ is fixed in \eref{power-model} (and, hence, in \eref{temperature-model}).
In other words, the parametrization $\vU(\o)$ is known, which we emphasize by $\vU$.
Assume for now that $\vP_\static$ in \eref{power-model} does not depend on $\vQ$, \ie, there is no interdependency between leakage and temperature.
Under the dynamic steady-state condition, $\vX_1 = \vX_{\nsteps + 1}$.
This boundary constraint and the recurrence in \eref{recurrence} yield a large system of $\nnodes \nsteps$ linear equations with $\nnodes \nsteps$ unknowns.
As shown in \cite{ukhov2012}, this problem can be efficiently solved by exploiting the particular structure of the system and the decomposition in \eref{eigenvalue-decomposition}.
The pseudocode of this algorithm is given in \aref{deterministic-solution}.\footnote{Hereafter, we utilize the MATLAB notation to refer to a particular column, \eg, $\m{A}(:, k)$, or a particular row, \eg, $\m{A}(k, :)$, of a matrix $\m{A}$. Also, we shall underline auxiliary vectors, \eg, $\tmp{\v{a}}$, and matrices, \eg, $\tmp{\m{A}}$.}
In this subsection, we address the case when $\o$ is fixed in \eref{power-model} (and, hence, in \eref{temperature-model}).
In other words, the parametrization $\vU(\o)$ is known, which we emphasize by $\vU$.
Assume for now that $\vP_\static$ in \eref{power-model} does not depend on $\vQ$, \ie, there is no interdependency between leakage and temperature.
Under the dynamic steady-state condition, $\vX_1 = \vX_{\nsteps + 1}$.
This boundary constraint and the recurrence in \eref{recurrence} yield a large system of $\nnodes \nsteps$ linear equations with $\nnodes \nsteps$ unknowns.
As shown in \cite{ukhov2012}, this problem can be efficiently solved by exploiting the particular structure of the system and the decomposition in \eref{eigenvalue-decomposition}.
The pseudocode of this algorithm is given in \aref{deterministic-solution}.\footnote{Hereafter, we utilize the MATLAB notation to refer to a particular column, \eg, $\m{A}(:, k)$, or a particular row, \eg, $\m{A}(k, :)$, of a matrix $\m{A}$. Also, we shall underline auxiliary vectors, \eg, $\tmp{\v{a}}$, and matrices, \eg, $\tmp{\m{A}}$.}
\input{include/algorithms/deterministic-solution.tex}

The time complexity of direct solutions of the system is $\bigO(\nsteps^3 \nnodes^3)$ while the one of \aref{deterministic-solution} is only $\bigO(\nsteps \nnodes^2 + \nnodes^3)$.\footnote{The complexity reported in \cite{ukhov2012} is erroneously overestimated to $\bigO(\nsteps \nnodes^3)$.}

In order to account for the leakage-temperature interdependence, \aref{deterministic-solution} is to be repeated for a sequence of total power profiles until the corresponding sequence of temperature profiles converges or some other stopping condition is met (\eg, a maximal temperature constraint is violated).
Each power profile is computed according to \eref{power-model} with respect to the temperature profile from the previous iteration starting from the ambient temperature.
This procedure is illustrated in \aref{deterministic-solution-with-leakage}.
\input{include/algorithms/deterministic-solution-with-leakage.tex}

In \aref{deterministic-solution-with-leakage}, $\mP_\static(\mQ, \vU)$ should be understood as a call to a subroutine that returns an $\nprocs \times \nsteps$ matrix wherein the $(i, j)$th element is the static component of the power dissipation of the $i$th processing element at the $j$th moment of time with respect to the temperature given by the $(i, j)$th element of $\mQ$ and $\vU$.


The time complexity of direct solutions of the system is $\bigO(\nsteps^3 \nnodes^3)$ while the one of \aref{deterministic-solution} is only $\bigO(\nsteps \nnodes^2 + \nnodes^3)$.\footnote{The complexity reported in \cite{ukhov2012} is erroneously overestimated to $\bigO(\nsteps \nnodes^3)$.}

In order to account for the leakage-temperature interdependence, \aref{deterministic-solution} is to be repeated for a sequence of total power profiles until the corresponding sequence of temperature profiles converges or some other stopping condition is met (\eg, a maximal temperature constraint is violated).
Each power profile is computed according to \eref{power-model} with respect to the temperature profile from the previous iteration starting from the ambient temperature.
This procedure is illustrated in \aref{deterministic-solution-with-leakage}.
\begin{algorithm}
  \caption{The deterministic \dssta\ considering the static power. \alab{deterministic-solution-with-leakage}}
  \begin{algorithmic}[1]
    \vspace{0.4em}

    \Require{$\mP_\dynamic \in \real^{\nprocs \times \nsteps}$ and $\vu$}
    \Ensure{$\mQ \in \real^{\nprocs \times \nsteps}$ and $\mP \in \real^{\nprocs \times \nsteps}$}

    \vspace{0.5em}

    \Let{$\mQ$}{$\mQ_\ambient$}

    \Repeat
      \Let{$\mP$}{$\mP_\dynamic + \mP_\static(\vu, \mQ)$}
      \Let{$\mQ$}{run {\bf Algorithm 1} for $\mP$}
    \Until{a stopping condition is satisfied}

    \vspace{0.4em}
  \end{algorithmic}
\end{algorithm}


In \aref{deterministic-solution-with-leakage}, $\mP_\static(\mQ, \vU)$ should be understood as a call to a subroutine that returns an $\nprocs \times \nsteps$ matrix wherein the $(i, j)$th element is the static component of the power dissipation of the $i$th processing element at the $j$th moment of time with respect to the temperature given by the $(i, j)$th element of $\mQ$ and $\vU$.


The time complexity of direct solutions of the system is $\bigO(\nsteps^3 \nnodes^3)$ while the one of \aref{deterministic-solution} is only $\bigO(\nsteps \nnodes^2 + \nnodes^3)$.\footnote{The complexity reported in \cite{ukhov2012} is erroneously overestimated to $\bigO(\nsteps \nnodes^3)$.}

In order to account for the leakage-temperature interdependence, \aref{deterministic-solution} is to be repeated for a sequence of total power profiles until the corresponding sequence of temperature profiles converges or some other stopping condition is met (\eg, a maximal temperature constraint is violated).
Each power profile is computed according to \eref{power-model} with respect to the temperature profile from the previous iteration starting from the ambient temperature.
This procedure is illustrated in \aref{deterministic-solution-with-leakage}.
\begin{algorithm}
  \caption{The deterministic \dssta\ considering the static power. \alab{deterministic-solution-with-leakage}}
  \begin{algorithmic}[1]
    \vspace{0.4em}

    \Require{$\mP_\dynamic \in \real^{\nprocs \times \nsteps}$ and $\vu$}
    \Ensure{$\mQ \in \real^{\nprocs \times \nsteps}$ and $\mP \in \real^{\nprocs \times \nsteps}$}

    \vspace{0.5em}

    \Let{$\mQ$}{$\mQ_\ambient$}

    \Repeat
      \Let{$\mP$}{$\mP_\dynamic + \mP_\static(\vu, \mQ)$}
      \Let{$\mQ$}{run {\bf Algorithm 1} for $\mP$}
    \Until{a stopping condition is satisfied}

    \vspace{0.4em}
  \end{algorithmic}
\end{algorithm}


In \aref{deterministic-solution-with-leakage}, $\mP_\static(\mQ, \vU)$ should be understood as a call to a subroutine that returns an $\nprocs \times \nsteps$ matrix wherein the $(i, j)$th element is the static component of the power dissipation of the $i$th processing element at the $j$th moment of time with respect to the temperature given by the $(i, j)$th element of $\mQ$ and $\vU$.


Hereafter, we shall adopt MATLAB's notations $\m{A}(k, :)$ and $\m{A}(:, k)$ to refer to the $k$th row and the $k$th column of a matrix $\m{A}$, respectively.
Also, auxiliary variables will be written with hats.
$\mQ_\ambient$ is a matrix of the ambient temperature.
\begin{remark}
The time complexity of direct solutions of the system of linear equations is $\complexity{\nsteps^3 \nnodes^3}$ while the one of \aref{deterministic-solution} is only $\complexity{\nsteps \nnodes^2 + \nnodes^3}$.
Note that the complexity of \aref{deterministic-solution} reported in \cite{ukhov2012} was erroneously overestimated.
\end{remark}

Let us now bring the leakage-temperature interdependence into the picture.
To this end, we repeat \aref{deterministic-solution} for a sequence of total power profiles $\{ \mP_k = \mP_\dynamic + \mP_{\static, k} \}$ wherein the static part $\mP_{\static, k}$ is being updated using \eref{power-model} given the temperature profile $\mQ_{k - 1}$ computed at the previous iteration starting from the ambient temperature.
The procedure stops when the sequence of temperature profiles $\{ \mQ_k \}$ converges in an appropriate norm, or some other stopping condition is satisfied (\eg, a maximal temperature constraint is violated).
This procedure is illustrated in \aref{deterministic-solution-with-leakage}.
\begin{algorithm}
  \caption{The deterministic \dssta\ considering the static power. \alab{deterministic-solution-with-leakage}}
  \begin{algorithmic}[1]
    \vspace{0.4em}

    \Require{$\mP_\dynamic \in \real^{\nprocs \times \nsteps}$ and $\vu$}
    \Ensure{$\mQ \in \real^{\nprocs \times \nsteps}$ and $\mP \in \real^{\nprocs \times \nsteps}$}

    \vspace{0.5em}

    \Let{$\mQ$}{$\mQ_\ambient$}

    \Repeat
      \Let{$\mP$}{$\mP_\dynamic + \mP_\static(\vu, \mQ)$}
      \Let{$\mQ$}{run {\bf Algorithm 1} for $\mP$}
    \Until{a stopping condition is satisfied}

    \vspace{0.4em}
  \end{algorithmic}
\end{algorithm}


In \aref{deterministic-solution-with-leakage}, $\mP_\static(\vu, \mQ)$ should be understood as a call to a subroutine that returns an $\nprocs \times \nsteps$ matrix wherein the $(i, k)$th element is the static component of the power dissipation of the $i$th processing element at the $k$th moment of time with respect to $\vu$ and the temperature given by the $(i, k)$th entry of $\mQ$.

\begin{remark}
A widespread approach to account for leakage is to linearize it with respect to temperature.
As shown in \cite{liu2007}, already one linear segment can deliver sufficiently accurate results.
One notable feature of such a linearization is that no looping-until-convergence is needed in this case; see \cite{ukhov2012}.
However, this technique assumes that the only varying parameter of leakage is temperature, and all other parameters have nominal values.
In that case, it is relatively easy to decide on a representative temperature range and undertake a one-dimensional curve-fitting procedure with respect to it.
In our case, the power model has multiple parameters stepping far from their nominal values, making it difficult to construct a good linear fit with respect to temperature.
\end{remark}
