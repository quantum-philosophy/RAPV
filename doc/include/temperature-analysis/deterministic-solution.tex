In this subsection, we address the case when $\o$ is fixed in \eref{power-model} (and, hence, in \eref{temperature-model}).
In other words, the parametrization $\vU(\o)$ is known, which we emphasize by $\vU$.
Assume for now that $\vP_\static$ in \eref{power-model} does not depend on $\vQ$, \ie, there is no interdependency between leakage and temperature.
Under the dynamic steady-state condition, $\vX_1 = \vX_{\nsteps + 1}$.
This boundary constraint and the recurrence in \eref{recurrence} yield a large system of $\nnodes \nsteps$ linear equations with $\nnodes \nsteps$ unknowns.
As shown in \cite{ukhov2012}, this problem can be efficiently solved by exploiting the particular structure of the system and the decomposition in \eref{eigenvalue-decomposition}.
The pseudocode of this algorithm is given in \aref{deterministic-solution}.\footnote{Hereafter, we utilize the MATLAB notation to refer to a particular column, \eg, $\m{A}(:, k)$, or a particular row, \eg, $\m{A}(k, :)$, of a matrix $\m{A}$. Also, we shall underline auxiliary vectors, \eg, $\tmp{\v{a}}$, and matrices, \eg, $\tmp{\m{A}}$.}
In this subsection, we address the case when $\o$ is fixed in \eref{power-model} (and, hence, in \eref{temperature-model}).
In other words, the parametrization $\vU(\o)$ is known, which we emphasize by $\vU$.
Assume for now that $\vP_\static$ in \eref{power-model} does not depend on $\vQ$, \ie, there is no interdependency between leakage and temperature.
Under the dynamic steady-state condition, $\vX_1 = \vX_{\nsteps + 1}$.
This boundary constraint and the recurrence in \eref{recurrence} yield a large system of $\nnodes \nsteps$ linear equations with $\nnodes \nsteps$ unknowns.
As shown in \cite{ukhov2012}, this problem can be efficiently solved by exploiting the particular structure of the system and the decomposition in \eref{eigenvalue-decomposition}.
The pseudocode of this algorithm is given in \aref{deterministic-solution}.\footnote{Hereafter, we utilize the MATLAB notation to refer to a particular column, \eg, $\m{A}(:, k)$, or a particular row, \eg, $\m{A}(k, :)$, of a matrix $\m{A}$. Also, we shall underline auxiliary vectors, \eg, $\tmp{\v{a}}$, and matrices, \eg, $\tmp{\m{A}}$.}
In this subsection, we address the case when $\o$ is fixed in \eref{power-model} (and, hence, in \eref{temperature-model}).
In other words, the parametrization $\vU(\o)$ is known, which we emphasize by $\vU$.
Assume for now that $\vP_\static$ in \eref{power-model} does not depend on $\vQ$, \ie, there is no interdependency between leakage and temperature.
Under the dynamic steady-state condition, $\vX_1 = \vX_{\nsteps + 1}$.
This boundary constraint and the recurrence in \eref{recurrence} yield a large system of $\nnodes \nsteps$ linear equations with $\nnodes \nsteps$ unknowns.
As shown in \cite{ukhov2012}, this problem can be efficiently solved by exploiting the particular structure of the system and the decomposition in \eref{eigenvalue-decomposition}.
The pseudocode of this algorithm is given in \aref{deterministic-solution}.\footnote{Hereafter, we utilize the MATLAB notation to refer to a particular column, \eg, $\m{A}(:, k)$, or a particular row, \eg, $\m{A}(k, :)$, of a matrix $\m{A}$. Also, we shall underline auxiliary vectors, \eg, $\tmp{\v{a}}$, and matrices, \eg, $\tmp{\m{A}}$.}
In this subsection, we address the case when $\o$ is fixed in \eref{power-model} (and, hence, in \eref{temperature-model}).
In other words, the parametrization $\vU(\o)$ is known, which we emphasize by $\vU$.
Assume for now that $\vP_\static$ in \eref{power-model} does not depend on $\vQ$, \ie, there is no interdependency between leakage and temperature.
Under the dynamic steady-state condition, $\vX_1 = \vX_{\nsteps + 1}$.
This boundary constraint and the recurrence in \eref{recurrence} yield a large system of $\nnodes \nsteps$ linear equations with $\nnodes \nsteps$ unknowns.
As shown in \cite{ukhov2012}, this problem can be efficiently solved by exploiting the particular structure of the system and the decomposition in \eref{eigenvalue-decomposition}.
The pseudocode of this algorithm is given in \aref{deterministic-solution}.\footnote{Hereafter, we utilize the MATLAB notation to refer to a particular column, \eg, $\m{A}(:, k)$, or a particular row, \eg, $\m{A}(k, :)$, of a matrix $\m{A}$. Also, we shall underline auxiliary vectors, \eg, $\tmp{\v{a}}$, and matrices, \eg, $\tmp{\m{A}}$.}
\input{include/algorithms/deterministic-solution.tex}

The time complexity of direct solutions of the system is $\bigO(\nsteps^3 \nnodes^3)$ while the one of \aref{deterministic-solution} is only $\bigO(\nsteps \nnodes^2 + \nnodes^3)$.\footnote{The complexity reported in \cite{ukhov2012} is erroneously overestimated to $\bigO(\nsteps \nnodes^3)$.}

In order to account for the leakage-temperature interdependence, \aref{deterministic-solution} is to be repeated for a sequence of total power profiles until the corresponding sequence of temperature profiles converges or some other stopping condition is met (\eg, a maximal temperature constraint is violated).
Each power profile is computed according to \eref{power-model} with respect to the temperature profile from the previous iteration starting from the ambient temperature.
This procedure is illustrated in \aref{deterministic-solution-with-leakage}.
\input{include/algorithms/deterministic-solution-with-leakage.tex}

In \aref{deterministic-solution-with-leakage}, $\mP_\static(\mQ, \vU)$ should be understood as a call to a subroutine that returns an $\nprocs \times \nsteps$ matrix wherein the $(i, j)$th element is the static component of the power dissipation of the $i$th processing element at the $j$th moment of time with respect to the temperature given by the $(i, j)$th element of $\mQ$ and $\vU$.


The time complexity of direct solutions of the system is $\bigO(\nsteps^3 \nnodes^3)$ while the one of \aref{deterministic-solution} is only $\bigO(\nsteps \nnodes^2 + \nnodes^3)$.\footnote{The complexity reported in \cite{ukhov2012} is erroneously overestimated to $\bigO(\nsteps \nnodes^3)$.}

In order to account for the leakage-temperature interdependence, \aref{deterministic-solution} is to be repeated for a sequence of total power profiles until the corresponding sequence of temperature profiles converges or some other stopping condition is met (\eg, a maximal temperature constraint is violated).
Each power profile is computed according to \eref{power-model} with respect to the temperature profile from the previous iteration starting from the ambient temperature.
This procedure is illustrated in \aref{deterministic-solution-with-leakage}.
\begin{algorithm}
  \caption{The deterministic \dssta\ considering the static power. \alab{deterministic-solution-with-leakage}}
  \begin{algorithmic}[1]
    \vspace{0.4em}

    \Require{$\mP_\dynamic \in \real^{\nprocs \times \nsteps}$ and $\vu$}
    \Ensure{$\mQ \in \real^{\nprocs \times \nsteps}$ and $\mP \in \real^{\nprocs \times \nsteps}$}

    \vspace{0.5em}

    \Let{$\mQ$}{$\mQ_\ambient$}

    \Repeat
      \Let{$\mP$}{$\mP_\dynamic + \mP_\static(\vu, \mQ)$}
      \Let{$\mQ$}{run {\bf Algorithm 1} for $\mP$}
    \Until{a stopping condition is satisfied}

    \vspace{0.4em}
  \end{algorithmic}
\end{algorithm}


In \aref{deterministic-solution-with-leakage}, $\mP_\static(\mQ, \vU)$ should be understood as a call to a subroutine that returns an $\nprocs \times \nsteps$ matrix wherein the $(i, j)$th element is the static component of the power dissipation of the $i$th processing element at the $j$th moment of time with respect to the temperature given by the $(i, j)$th element of $\mQ$ and $\vU$.


The time complexity of direct solutions of the system is $\bigO(\nsteps^3 \nnodes^3)$ while the one of \aref{deterministic-solution} is only $\bigO(\nsteps \nnodes^2 + \nnodes^3)$.\footnote{The complexity reported in \cite{ukhov2012} is erroneously overestimated to $\bigO(\nsteps \nnodes^3)$.}

In order to account for the leakage-temperature interdependence, \aref{deterministic-solution} is to be repeated for a sequence of total power profiles until the corresponding sequence of temperature profiles converges or some other stopping condition is met (\eg, a maximal temperature constraint is violated).
Each power profile is computed according to \eref{power-model} with respect to the temperature profile from the previous iteration starting from the ambient temperature.
This procedure is illustrated in \aref{deterministic-solution-with-leakage}.
\begin{algorithm}
  \caption{The deterministic \dssta\ considering the static power. \alab{deterministic-solution-with-leakage}}
  \begin{algorithmic}[1]
    \vspace{0.4em}

    \Require{$\mP_\dynamic \in \real^{\nprocs \times \nsteps}$ and $\vu$}
    \Ensure{$\mQ \in \real^{\nprocs \times \nsteps}$ and $\mP \in \real^{\nprocs \times \nsteps}$}

    \vspace{0.5em}

    \Let{$\mQ$}{$\mQ_\ambient$}

    \Repeat
      \Let{$\mP$}{$\mP_\dynamic + \mP_\static(\vu, \mQ)$}
      \Let{$\mQ$}{run {\bf Algorithm 1} for $\mP$}
    \Until{a stopping condition is satisfied}

    \vspace{0.4em}
  \end{algorithmic}
\end{algorithm}


In \aref{deterministic-solution-with-leakage}, $\mP_\static(\mQ, \vU)$ should be understood as a call to a subroutine that returns an $\nprocs \times \nsteps$ matrix wherein the $(i, j)$th element is the static component of the power dissipation of the $i$th processing element at the $j$th moment of time with respect to the temperature given by the $(i, j)$th element of $\mQ$ and $\vU$.


The time complexity of direct solutions of the system is $\bigO(\nsteps^3 \nnodes^3)$ while the one of \aref{deterministic-solution} is only $\bigO(\nsteps \nnodes^2 + \nnodes^3)$.\footnote{The complexity reported in \cite{ukhov2012} is erroneously overestimated to $\bigO(\nsteps \nnodes^3)$.}

In order to account for the leakage-temperature interdependence, \aref{deterministic-solution} is to be repeated for a sequence of total power profiles until the corresponding sequence of temperature profiles converges or some other stopping condition is met (\eg, a maximal temperature constraint is violated).
Each power profile is computed according to \eref{power-model} with respect to the temperature profile from the previous iteration starting from the ambient temperature.
This procedure is illustrated in \aref{deterministic-solution-with-leakage}.
\begin{algorithm}
  \caption{The deterministic \dssta\ considering the static power. \alab{deterministic-solution-with-leakage}}
  \begin{algorithmic}[1]
    \vspace{0.4em}

    \Require{$\mP_\dynamic \in \real^{\nprocs \times \nsteps}$ and $\vu$}
    \Ensure{$\mQ \in \real^{\nprocs \times \nsteps}$ and $\mP \in \real^{\nprocs \times \nsteps}$}

    \vspace{0.5em}

    \Let{$\mQ$}{$\mQ_\ambient$}

    \Repeat
      \Let{$\mP$}{$\mP_\dynamic + \mP_\static(\vu, \mQ)$}
      \Let{$\mQ$}{run {\bf Algorithm 1} for $\mP$}
    \Until{a stopping condition is satisfied}

    \vspace{0.4em}
  \end{algorithmic}
\end{algorithm}


In \aref{deterministic-solution-with-leakage}, $\mP_\static(\mQ, \vU)$ should be understood as a call to a subroutine that returns an $\nprocs \times \nsteps$ matrix wherein the $(i, j)$th element is the static component of the power dissipation of the $i$th processing element at the $j$th moment of time with respect to the temperature given by the $(i, j)$th element of $\mQ$ and $\vU$.
