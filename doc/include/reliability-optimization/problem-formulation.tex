Consider a periodic application which is composed of a number of tasks and is given as a directed acyclic graph.
The graph has $\ntasks$ vertices representing the tasks and a number of edges specifying data dependencies between those tasks.
Any processing element can execute any task, and each pair of a processing element and a task is characterized by an execution time and dynamic power.
Since the proposed techniques are orientated towards the design stage, static scheduling is considered, which is typically done offline.
More specifically, the application is scheduled using a static cyclic scheduler, and schedules are generated using the list scheduling policy \cite{adam1974}.
A schedule is defined as a mapping of the tasks onto the processing elements and the corresponding starting times; we shall denote it by $\schedule$.
The goal of our optimization is to find such a schedule $\schedule$ that minimizes the energy consumption while satisfying certain constraints.

Since energy is a function of power, and power depends on a set of uncertain parameters, the energy consumption is a \rv\ at the design stage, which we denote by $\qoiE$.
Our objective is to minimize the expected value of $\qoiE$:
\begin{equation} \elab{objective}
  \min_{\schedule} \expectation{\qoiE(\schedule)}
\end{equation}
where
\[
  \qoiE(\schedule) = \dt \sum \mP(\schedule),
\]
$\dt$ is the sampling interval of the power profile $\mP$, and $\sum \mP$ denotes the summation over all elements of $\mP$.
Hereafter, we also emphasize the dependency on $\schedule$.
Our constraints are (i) time, (ii) temperature, and (iii) reliability as follows.
(i) The period of the application is constrained by $\t_\maximal$ (a deadline).
(ii) The maximal temperature that the system can tolerate is constrained by $\q_\maximal$, and $\pr_\burn$ is an acceptable probability of burning the chip.
(iii) The minimal time that the system should survive is constrained by $\T_\minimal$, and $\pr_\wear$ is an acceptable probability of having a premature fault due to wear.
The three constraints are formalized as follows:
\begin{align}
  & \period(\schedule) \leq \t_\maximal, \elab{timing-constraint} \\
  & \probabilityMeasure\left( \qoiQ(\schedule) \geq \q_\maximal \right) \leq \pr_\burn, \text{ and} \elab{thermal-constraint} \\
  & \probabilityMeasure\left( \qoiT(\schedule) \leq \T_\minimal \right) \leq \pr_\wear. \elab{reliability-constraint}
\end{align}
In \eref{timing-constraint}--\eref{reliability-constraint}, $\period$ is the period of the application according to the schedule,
\begin{align*}
  & \qoiQ(\schedule) = \norm[\infty]{\mQ(\schedule)}, \\
  & \qoiT(\schedule) = \expectation{\T(\schedule) \, | \, \eta} = \eta(\schedule) \, \Gamma\left(1 + \frac{1}{\beta}\right), \text{ and}
\end{align*}
$\norm[\infty]{\mQ}$ denotes the extraction of the maximal value from the temperature profile $\mQ$.
The last two constraints, \ie, \eref{thermal-constraint} and \eref{reliability-constraint}, are probabilistic as the quantities under consideration are random.
In \eref{reliability-constraint}, we set an upper bound on the probability of the expected value of $\T$, and it is important to realize that this expectation is a \rv\ itself due to the nested structure of the reliability model described in \rref{two-level-probabilistic-modeling}.
