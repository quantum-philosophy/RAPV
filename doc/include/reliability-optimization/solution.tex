In order to evaluate \eref{objective}--\eref{reliability-constraint}, we utilize the uncertainty analysis technique presented in \sref{uncertainty-analysis}.
In this case, the quantity of interest is a vector with three elements:
\begin{equation} \elab{quantity-of-interest}
  \vw = (\qoiE, \qoiQ, \qoiT).
\end{equation}
Although it is not spelled out, each quantity depends on $\schedule$.
The first element corresponds to the energy consumption used in \eref{objective}, the second element is the maximal temperature used in \eref{thermal-constraint}, and the last one is the scale parameter of the reliability model (see \sref{reliability-analysis}) used in \eref{reliability-constraint}.
The uncertainty analysis in \sref{uncertainty-analysis} should be applied as explained in \rref{multiple-dimensions}.
In \aref{surrogate-construction}, Algorithm~X is an intermediate procedure that makes a call to \aref{temperature-solution} and processes the resulting power and temperature profiles as required by \eref{quantity-of-interest}.

We use a genetic algorithm for optimization.
Each chromosome is a $2 \ntasks$-element vector (twice the number of tasks) concatenating a pair of two vectors.
The first is a vector in $\{ 1, 2, \dotsc, \nprocs \}^\ntasks$ that maps the tasks onto the processing elements (\ie, a mapping).
The second is a vector in $\{ 1, 2, \dotsc, \ntasks \}^\ntasks$ that orders the tasks according to their priorities (\ie, a ranking).
Since we rely on a static cyclic scheduler and the list scheduling policy, such a pair of vectors uniquely encodes a schedule $\schedule$.
The population contains $4 \ntasks$ individuals which are initialized using uniform distributions.
The parents for the next generation are chosen by a tournament selection with the number of competitors equal to 20\% of $\ntasks$.
A one-point crossover is then applied to 80\% of the parents.
Each parent undergoes a uniform mutation wherein each gene is altered with probability 0.01.
The top five-percent individuals always survive.
The stopping condition is the absence of improvement within 10 successive generations.

Let us turn to the evaluation of a chromosome's fitness.
We begin by checking the timing constraint given in \eref{timing-constraint} as it does not require any probabilistic analysis; the constraint is purely deterministic.
If \eref{timing-constraint} is violated, we set the fitness to the amount of this violation relative to the constraint---that is, to the difference between the actual application period and the deadline $\t_\maximal$ divided by $\t_\maximal$---and add a large constant, say, $C$, on top.
If \eref{timing-constraint} is satisfied, we perform our probabilistic analysis and proceed to checking the constraints in \eref{thermal-constraint} and \eref{reliability-constraint}.
If any of the two is violated, we set the fitness to the total relative amount of violation plus $C/2$.
If all the constraints are satisfied, the fitness value of the chromosome is set to the expected consumption of energy, as in shown in \eref{objective}.

In order to speed up the optimization, we make use of caching and parallel computing.
Specifically, the fitness value of each evaluated chromosome is stored in memory and pulled out when a chromosome with the same set of genes is encountered, and unseen (not cached) individuals are assessed in parallel.
