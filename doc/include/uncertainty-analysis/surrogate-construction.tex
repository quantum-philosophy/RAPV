Let $\w: \outcomes \to \real$ be a quantity of interest dependent on $\vu$.
For convenience, $\w$ is assumed to be one-dimension, which will be generalized later on.
In order to give a computationally efficient probabilistic characterization of $\w$, we utilize non-intrusive spectral decompositions based on orthogonal polynomials.
The corresponding mathematical foundation is outlined in \xref{spectral-decomposition} and \xref{numerical-integration}, and here we leap directly to the main results obtained in those sections.

\subsubsection{Classical decomposition}
Assume that $\w \in \L{2}(\outcomes, \sigmaAlgebra, \probabilityMeasure)$.
Then $\w$ can be expanded into the following series:
\begin{equation} \elab{spectral-decomposition}
  \w \approx \chaos{\nvars}{\nclevel}{\w} := \sum_{\multiindex \in \chaosMultiindexSet{\nclevel}} \coefficient{\w}_{\multiindex} \, \polynomial_{\multiindex}(\vz)
\end{equation}
where $\nclevel$ is the expansion level; $\multiindex = (\alpha_i) \in \natural^\nvars$ is a multi-index; $\chaosMultiindexSet{\nclevel}$ is an index set to be discussed shortly; and $\polynomial_{\multiindex}(\vz)$ is an $\nvars$-variate Hermite polynomial constructed as a product of normalized one-dimensional Hermite polynomials of orders specified by the corresponding elements of $\multiindex$.

As discussed in \xref{spectral-decomposition}, each coefficient $\coefficient{\w}_{\multiindex}$ in \eref{spectral-decomposition} is an $\nvars$-dimensional integral of the product of $\w$ with $\polynomial_{\multiindex}$, and this integral should be computed numerically.
To this end, we construct a quadrature rule and calculate $\coefficient{\w}_{\multiindex}$ as
\begin{equation} \elab{numerical-integration}
  \coefficient{\w}_{\multiindex} \approx \quadrature{\nvars}{\nqlevel}{\w \, \polynomial_{\multiindex}} := \sum_{i = 1}^\nqorder \w(\transformation{\point_i}) \, \polynomial_{\multiindex}(\point_i) \, \weight_i
\end{equation}
where $\nqlevel$ is the quadrature level, and $\{ (\point_i \in \real^\nvars, \weight_i \in \real) \}_{i = 1}^\nqorder$ are the points and weights of the quadrature.
The multivariate quadrature operator $\quadrature$ is based on a set of univariate operators and is constructed as follows:
\begin{equation} \elab{smolyak-sparse-grid}
  \quadrature = \bigoplus_{\multiindex \in \quadratureMultiindexSet{\nqlevel}} \Delta_{\alpha_1} \otimes \cdots \otimes \Delta_{\alpha_\nvars}.
\end{equation}
The notation used in the above equation is not essential for the present discussion and can be found in \xref{numerical-integration}.
The important aspect to note, though, is the structure of this operator, namely, the index set $\quadratureMultiindexSet{\nqlevel}$.

The standard choice of $\multiindexSet{\nclevel}$ in \eref{spectral-decomposition} is $\{ \multiindex: \norm[1]{\multiindex} \leq \nclevel \}$, which is called an isotropic total-order index set.
\emph{Isotropic} refers to the fact that all dimensions are trimmed identically, and \emph{total-order} to the structure of the corresponding polynomial space.
In \eref{numerical-integration}, $\polynomial_{\multiindex}$ is a polynomial of total order at most $\nclevel$, and $\w$ is modeled as such a polynomial.
Hence, the integrand in \eref{numerical-integration} is a polynomial of total order at most $2 \nclevel$.
Having this aspect in mind, one usually constructs a quadrature rule such that it is exact for polynomials of total order $2 \nclevel$ \cite{eldred2008}.
In this work, we employ Gaussian quadratures for integration, in which case a quadrature of level $\nqlevel$ is exact for integrating polynomials of total order $2 \nqlevel + 1$ (see \xref{numerical-integration}).
Therefore, it is sufficient to keep $\nclevel$ and $\nqlevel$ equal.
More generally, the index sets $\multiindexSet{\nclevel}$ and $\multiindexSet{\nqlevel}$ should be synchronized; in what follows, we shall denote both by $\multiindexSet{\level}$.

\subsubsection{Anisotropic decomposition}
In the context of sparse grids, an important generalization of the construction in \eref{smolyak-sparse-grid} is the so-called anisotropic Smolyak algorithm \cite{nobile2008}.
The main difference between the isotropic and anisotropic versions lies in the constraints imposed on $\quadratureMultiindexSet{\level}$.
An anisotropic total-order index set is as follows:
\begin{equation} \elab{anisotropic-total-order-index-set}
  \chaosMultiindexSet{\level} = \left\{ \multiindex: \innerProduct{\dimensionAnisotropy, \multiindex} \leq \level \, \min_i c_i \right\}
\end{equation}
where $\dimensionAnisotropy = (c_i) \in \real^\nvars$, $c_i \geq 0$, is a vector assigning importance coefficients to each dimension, and $\innerProduct{\cdot, \cdot}$ is the standard inner product on $\real^\nvars$.
Equation \eref{anisotropic-total-order-index-set} plugged into \eref{smolyak-sparse-grid} results in a sparse grid which is exact for the polynomial space which is tailored using the same index set.

The above approach allows one to leverage the highly anisotropic behaviors inherent for many practical problems \cite{nobile2008}.
It provides a great control over the computational time associated with the construction of spectral decompositions: a carefully chosen importance vector $\dimensionAnisotropy$ in \eref{anisotropic-total-order-index-set} can significantly reduce the number of polynomial terms in \eref{spectral-decomposition} and the number of quadrature points needed in \eref{numerical-integration} to compute the coefficients of those terms.
The question to discuss now is the choice of $\dimensionAnisotropy$.
In this regard, we rely on the variance contributions of the dimensions given by  $\dimensionContribution$ in \eref{dimension-contribution}.
Specifically, we let
\begin{equation} \elab{dimension-anisotropy}
  \dimensionAnisotropy = \dimensionContribution^\anisotropyKnob := (\lambda_i^\anisotropyKnob)_{i = 1}^\nvars
\end{equation}
where $\anisotropyKnob \in [0, 1]$ is a tuning parameter.
The isotropic scenario can be recovered by setting $\anisotropyKnob = 0$; the other values of $\anisotropyKnob$ correspond to various levels of anisotropy with the maximum attained by setting $\anisotropyKnob = 1$.

Let us sum up what we have at this point.
In order to give a probabilistic characterization of a quantity of interest, we perform chaotic expansions as shown in \eref{spectral-decomposition}.
The coefficients of such expansions are evaluated by Gaussian quadratures as shown in \eref{numerical-integration}.
The quadratures are constructed using the Smolyak formula given in \eref{smolyak-sparse-grid}.
The index sets used in both \eref{spectral-decomposition} and \eref{smolyak-sparse-grid} are the one given in \eref{anisotropic-total-order-index-set} wherein the anisotropic weights are set according to \eref{dimension-contribution} and \eref{dimension-anisotropy}.

\subsubsection{Efficient implementation}
The pair of $\vz$ and $\dimensionAnisotropy$ uniquely characterizes the uncertainty quantification problem at hand.
Once these two components have been identified, and the desired approximation level $\level = \nclevel = \nqlevel$ has been specified, the corresponding polynomial basis and quadrature stay the same for all quantities that one might be interested in studying.
This observation is of high importance as a lot of preparatory work can and should be done only once and then stored/tabulated for future uses.
In particular, the construction in \eref{spectral-decomposition} can be reduced one multiplication with a precomputed matrix as follows.

Let $\ncorder = \cardinality{\multiindexSet{\level}}$ be the cardinality of $\multiindexSet{\level}$, which is also the number of polynomial terms and, hence, coefficients in \eref{spectral-decomposition}.
Assume the multi-indices contained in $\multiindexSet{\level}$ are arranged in a vector $(\multiindex_i)_{i = 1}^\ncorder$, which gives a certain ordering.
Now, let
\begin{equation} \elab{projection-matrix}
  \projection = \big( \pi_{ij} = \polynomial_{\multiindex_i}(\point_j) \, \weight_j \big)_{i = 1, \, j = 1}^{i = \ncorder, \, j = \nqorder},
\end{equation}
that is, $\pi_{ij}$ is the polynomial corresponding to the $i$th multi-index evaluated at the $j$th quadrature point and multiplied by the $j$th quadrature weight.
We refer to $\projection$ as the projection matrix.
The coefficients in \eref{spectral-decomposition} can now be computed as
\begin{equation} \elab{coefficient-evaluation}
  \coefficient{\vw} = \projection \, \vw
\end{equation}
where
\begin{equation} \elab{quantity-evaluation}
  \coefficient{\vw} = (\coefficient{\w}_i)_{i = 1}^\ncorder \hspace{1em} \text{and} \hspace{1em} \vw = \big( \w_i = \w(\transformation{\point_i}) \big)_{i = 1}^\nqorder,
\end{equation}
that is, $\w_i$ is $\w$ evaluated at the $i$th quadrature node.
It can be seen that \eref{coefficient-evaluation} is just a matrix version of \eref{numerical-integration}.
$\projection$ is the one that should be precomputed and tabulated.
The pseudocode of the procedure is given in \aref{surrogate-construction} wherein Algorithm~X stands for the routine that calculates $\w$ for a given $\vu$.
\begin{algorithm}
  \caption{Surrogate construction \alab{surrogate-construction}}
  \begin{algorithmic}[1]
    \vspace{0.4em}

    \Require{Algorithm X} \Comment{the subroutine evaluating $\w$}
    \Ensure{$\coefficient{\mW} \in \real^\ncorder$} \Comment{the expansion coefficients}

    \vspace{0.5em}

    \For{$i \gets 1 \textrm{ to } \nqorder$} \Comment{for each quadrature point $\point_i$}
      \Let{$\vu$}{$\transformation{\point_i}$}
      \Let{$\mW(i)$}{call {\bf Algorithm X} for $\vu$}
    \EndFor

    \Let{$\coefficient{\mW}$}{$\projection \, \mW$}

    \vspace{0.4em}
  \end{algorithmic}
\end{algorithm}
\vspace{-0.5em}


\begin{remark} \rlab{multiple-dimensions}
When $\w$ is multidimensional, we shall consider it as a row vector with an appropriate number of elements.
Then all the operation with respect to $\w$, such as those in \eref{spectral-decomposition} and \eref{numerical-integration}, should be undertaken elementwise.
In \eref{coefficient-evaluation}, \eref{quantity-evaluation}, and \aref{surrogate-construction}, $\vw$ and $\coefficient{\vw}$ are to be treated as matrices with $\ncorder$ rows, and $\w_i$ as a row vector.
The output of Algorithm~X is assumed to be automatically reshaped into a row vector.
\end{remark}
