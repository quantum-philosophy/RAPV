Let $\w \equiv \w(\vu) \equiv \w(\transformation{\vz})$ be a quantity of interest dependent on $\vu$.
In order to give a computationally efficient probabilistic characterization of $\w$, we utilize non-intrusive spectral decompositions based on orthogonal polynomials.
The corresponding mathematical foundation is outlined in \xref{spectral-decomposition} and \xref{numerical-integration}, and here we leap directly to the main results obtained in those sections.

Assuming $\w \in \L{2}(\outcomes, \sigmaAlgebra, \probabilityMeasure)$, \ie, the variance of $\w$ is finite.
Then $\w$ can be expanded into the following series:
\begin{equation} \elab{spectral-decomposition}
  \w \approx \chaos{\nvars}{\nclevel}{\w} := \sum_{\multiindex \in \chaosMultiindexSet{\nclevel}} \coefficient{\w}_{\multiindex} \, \polynomial_{\multiindex}(\vz)
\end{equation}
where $\nclevel$ is the expansion level; $\multiindex = (\alpha_i) \in \natural^\nvars$ is a multi-index; $\chaosMultiindexSet{\nclevel}$ is an index set to be discussed shortly; and $\polynomial_{\multiindex}(\vz)$ is an $\nvars$-variate Hermite polynomial constructed as a product of normalized one-dimensional Hermite polynomials of orders specified by the corresponding elements of $\multiindex$.

As discussed in \xref{spectral-decomposition}, each coefficient $\coefficient{\w}_{\multiindex}$ in \eref{spectral-decomposition} is an $\nvars$-dimensional integral of the product of $\w$ with $\polynomial_{\multiindex}$, and this integral should be computed numerically.
To this end, we construct a quadrature rule and calculate $\coefficient{\w}_{\multiindex}$ as
\begin{equation} \elab{numerical-integration}
  \coefficient{\w}_{\multiindex} \approx \quadrature{\nvars}{\nqlevel}{\w \, \polynomial_{\multiindex}} := \sum_{i = 1}^\nqorder \w(\transformation{\point_i}) \, \polynomial_{\multiindex}(\point_i) \, \weight_i
\end{equation}
where $\nqlevel$ is the quadrature level, and $\{ (\point_i \in \real^\nvars, \weight_i \in \real) \}_{i = 1}^\nqorder$ are the points and weights of the quadrature.
The multivariate quadrature operator $\quadrature$ is based on a set of univariate operators and is constructed as follows:
\begin{equation} \elab{smolyak-sparse-grid}
  \quadrature = \bigoplus_{\multiindex \in \quadratureMultiindexSet{\nqlevel}} \Delta_{\alpha_1} \otimes \cdots \otimes \Delta_{\alpha_\nvars}.
\end{equation}
The notation used in the above equation is not essential for the present discussion and can be found in \xref{numerical-integration}.
The important aspect to note, though, is the structure of this operator, namely, the index set $\quadratureMultiindexSet{\nqlevel}$.

The standard choice of $\multiindexSet{\nclevel}$ in \eref{spectral-decomposition} is $\{ \multiindex: \norm[1]{\multiindex} \leq \nclevel \}$ where $\norm[1]{\multiindex} := \sum_i |\alpha_i|$, which is called an isotropic total-order index set.
\emph{Isotropic} refers to the fact that all dimensions are trimmed identically, and \emph{total-order} to the structure of the corresponding polynomial space.
In \eref{numerical-integration}, $\polynomial_{\multiindex}$ is a polynomial of total order at most $\nclevel$, and $\w$ is modeled as such a polynomial.
Hence, the integrand in \eref{numerical-integration} is a polynomial of total order at most $2 \nclevel$.
Keeping this aspect in mind, one usually chooses a quadrature rule such that the rule is exact for polynomials of total order $2 \nclevel$ \cite{eldred2008}.
In this work, we employ Gaussian quadratures for integration, in which case a quadrature of level $\nqlevel$ is exact for integrating polynomials of total order $2 \nqlevel + 1$.
Hence, it is sufficient to keep $\nclevel$ and $\nqlevel$ equal.
More generally, the index sets $\multiindexSet{\nclevel}$ and $\multiindexSet{\nqlevel}$ should be synchronized; in what follows, we shall denote both by $\multiindexSet{\level}$.

In the context of sparse grids, an important generalization of the construction in \eref{smolyak-sparse-grid} is the so-called anisotropic Smolyak algorithm \cite{nobile2008}.
The main difference between the isotropic and anisotropic versions lies in the constraints imposed on $\quadratureMultiindexSet{\level}$.
An anisotropic total-order index set is
\begin{equation} \elab{anisotropic-total-order-index-set}
  \chaosMultiindexSet{\level} = \left\{ \multiindex: \: \innerProduct{\multiindexWeight, \multiindex} \leq \nclevel \, \min_i c_i \right\}
\end{equation}
where $\multiindexWeight = (c_i) \in \real^\nvars_+$ is a vector assigning weights to each dimension, and $\innerProduct{\cdot, \cdot}$ is the standard inner product on $\real^\nvars$.
The index set in \eref{anisotropic-total-order-index-set} plugged into \eref{smolyak-sparse-grid} results in a sparse grid which is exact for polynomials constrained using the same index set.
Consequently, we use \eref{anisotropic-total-order-index-set} in both \eref{spectral-decomposition} and \eref{numerical-integration} (via \eref{smolyak-sparse-grid}).
