Our foremost task now is to transform $\vu$ into mutually independent \rvs\ as independence is essential for the forthcoming mathematical treatment and practical computations.
To this end, an adequate probability transformation should be undertaken depending on the available information; see \cite{eldred2008} for an overview.
One transformation for which the assumed knowledge about $\vu$ is sufficient is the Nataf transformation \cite{li2008}.
Denote this transformation by
\begin{equation} \elab{probability-transformation}
  \vu = \transformation{\vz},
\end{equation}
which relates $\nparams \nprocs$ dependent \rvs, \ie, $\vu$, with $\nvars$ independent \rvs
\begin{equation} \elab{independent-random-variables}
  \vz = (\z_i)_{i = 1}^\nvars.
\end{equation}
Regardless of the marginals, $\z_i \sim \gaussianDistribution{0, 1}$, $i = 1, \dots, \nvars$, that is, each $\z_i$ has the standard Gaussian distribution.
Refer to \xref{probability-transformation} for further details about the Nataf transformation.

As we shall discuss later on, the stochastic dimensionality $\nvars$ of the problem has a considerable impact of the computational complexity of our framework.
Therefore, an important part of the preprocessing stage is model order reduction.
To this end, we preserve only those stochastic dimensions whose contribution to the total variance of $\vu$ is the most significant, which is identified by the eigenvalues of $\mCorr_\u$:
\begin{equation} \elab{dimension-contribution}
  \dimensionContribution = (\lambda_i)_{i = 1}^{\nparams \nprocs}, \hspace{1em} \norm[1]{\dimensionContribution} = 1,
\end{equation}
as it is further discussed in \xref{model-order-reduction}.
Without introducing additional transformations, we let $\transformation$ be augmented with such a reduction procedure.
Thus, we have that $\nvars \leq \nparams \nprocs$.

Let us turn to the illustrative application.
Recall that we exemplify our framework considering the effective channel length and gate-oxide thickness with the notation given in \eref{application-uncertain-parameters}.
Both parameters correspond to Euclidean distances; they take values on bounded intervals of the positive part of the real line.
With this in mind, we model the two process parameters using the four-parametric family of beta distributions:
\begin{equation*}
  \u_i \sim \distribution_{\u_i} = \betaDistribution{a_i, b_i, c_i, d_i}
\end{equation*}
where $i = 1, \dots, 2 \nprocs$, $a_i$ and $b_i$ control the shape of the distributions, and $[ c_i, d_i ]$ correspond their supports.
Without loss of generality, we let the two considered process parameters be independent of each other, and the correlations among those elements of $\vu$ that correspond to the same process parameter be given by the following correlation function:
\begin{equation} \elab{correlation-function}
  \fCorr(\vr_1, \vr_2) = \sum_i \varpi_i \; \fCorr_i(\vr_1, \vr_2)
\end{equation}
where $\{ \fCorr_i: \domain \times \domain \to \real \}$ are a number of kernel functions, and $\bm{\varpi} = (\varpi_i)$ are their weights such that $\varpi_i \geq 0$ and $\norm[1]{\bm{\varpi}} = 1$.
The entries of $\mCorr_\u$ are computed using \eref{correlation-function}.
Each kernel $\fCorr_i$ captures certain correlation features inherent to the fabrication process at hand.
In this regard, we shall use the same setup as the one utilized in \cite{ukhov2014} with equal weights (the squared-exponential and Ornstein--Uhlenbeck kernels).
