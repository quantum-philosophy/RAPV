Our foremost task now is to transform $\vu$ into mutually independent \rvs\ as independence is essential for the forthcoming mathematical treatment and practical computations.
To this end, an adequate probability transformation should be undertaken depending on the available information; see \cite{eldred2008} for an overview.
One transformation for which the assumed knowledge about $\vu$ is sufficient is the Nataf transformation \cite{li2008}.
Denote this transformation by
\begin{equation} \elab{probability-transformation}
  \vu = \transformation{\vz},
\end{equation}
which relates $\nparams \nprocs$ dependent \rvs, \ie, $\vu$, with $\nvars = \nparams \nprocs$ independent \rvs
\begin{equation} \elab{independent-random-variables}
  \vz = (\z_i)_{i = 1}^\nvars.
\end{equation}
Regardless of the marginals, $\z_i \sim \gaussianDistribution{0, 1}$, $i = 1, 2, \dots, \nvars$, that is, each $\z_i$ has the standard Gaussian distribution.
Refer to \xref{probability-transformation} for further details about the Nataf transformation.

As we shall discuss later on, the stochastic dimensionality $\nvars$ has a considerable impact on the computational complexity of our framework.
Therefore, an important part of the preprocessing stage is model order reduction.
To this end, we preserve only those stochastic dimensions whose contribution to the total variance of $\vu$ is the most significant, which is identified by the eigenvalues of the correlation matrix $\mCorr_\u$:
\begin{equation} \elab{dimension-contribution}
  \dimensionContribution = (\lambda_i)_{i = 1}^{\nparams \nprocs}, \hspace{1em} \norm[1]{\dimensionContribution} = 1,
\end{equation}
as it is further discussed in \xref{model-order-reduction}.
Without introducing additional transformations, we let $\transformation$ in \eref{probability-transformation} be augmented with such a reduction procedure and redefine $\vz \in \real^\nvars$ as the reduced independent random variables where $\nvars \leq \nparams \nprocs$.

Let us turn to the illustrative application.
Recall that we exemplify our framework considering the effective channel length and gate-oxide thickness with the notation given in \eref{application-uncertain-parameters}.
Both parameters correspond to Euclidean distances; they take values on bounded intervals of the positive part of the real line.
With this in mind, we model the two process parameters using the four-parametric family of beta distributions:
\begin{equation*}
  \u_i \sim \distribution_{\u_i} = \betaDistribution{a_i, b_i, c_i, d_i}
\end{equation*}
where $i = 1, 2, \dots, 2 \nprocs$, $a_i$ and $b_i$ control the shape of the distributions, and $[ c_i, d_i ]$ correspond to their supports.
Without loss of generality, we let the two considered process parameters be independent of each other, and the correlations among those elements of $\vu$ that correspond to the same process parameter be given by the following correlation function:
\begin{equation} \elab{correlation-function}
  \fCorr(\vr_i, \vr_j) = \varpi \; \fCorrSE(\vr_i, \vr_j) + (1 - \varpi) \fCorrOU(\vr_i, \vr_j)
\end{equation}
where $\vr_i \in \real^2$ is the center of the $i$th processing element relative to the center of the die. The correlation function is a composition of two kernels:
\begin{align*}
  & \fCorrSE(\vr_i, \vr_j) = \exp\left(- \frac{\norm{\vr_i - \vr_j}^2}{\lCorrSE^2} \right) \text{ and} \\
  & \fCorrOU(\vr_i, \vr_j) = \exp\left(- \frac{\abs{\,\norm{\vr_i} - \norm{\vr_j}\,}}{\lCorrOU} \right),
\end{align*}
which are known as the squared-exponential and Ornstein--Uhlenbeck kernels, respectively.
In the above formulae, $\varpi \in [0, 1]$ is a weight coefficient balancing the kernels; $\lCorrSE$ and $\lCorrOU > 0$ are so-called length-scale parameters; and $\norm{\cdot}$ stands for the Euclidean norm in $\real^2$.
The choice of these two kernels is guided by the observations of the correlation patterns induced by the fabrication process: $\fCorrSE$ imposes similarities between those spatial locations that are close to each other, and $\fCorrOU$ imposes similarities between those locations that are at the same distance from the center of the die; see, \eg, \cite{friedberg2005} for additional details.
The length-scale parameters $\lCorrSE$ and $\lCorrOU$ control the extend of these similarities, \ie, the range wherein the influence of one point on another is significant.
