Let us now apply the proposed analysis techniques in the context of design-space exploration.
Assume that the periodic application under consideration is composed of a number of tasks and is given as a directed acyclic graph.
The graph has $\ntasks$ vertices representing the tasks, and the edges of the graph specify data dependencies between those tasks.
Any processing element can execute any task, and each potential pair (a processing element and a task) is characterized by an execution time and dynamic power consumption.
The tasks are scheduled onto the processing elements using a list scheduler.
The scheduler constructs a schedule using the information provided by two vectors: a mapping vector $\mapping \in \{ 1, 2, \dotsc, \nprocs \}^\ntasks$ and a ranking vector $\priority \in \{ 1, 2, \dotsc, \ntasks \}^\ntasks$.
The former is an assignment of the tasks to the processing elements, and the latter is an ordering of the tasks according to their priorities.
The goal of our optimization is to find such a pair of $\mapping$ and $\priority$ that minimizes the energy consumed by the system while satisfying certain constraints.

Since energy is a function of power, and power depends on a set of uncertain parameters, the energy consumption is a \rv\ at the design stage, which we denote by $\qoiE$.
Our objective is then to minimize the expected value of $\qoiE$:
\begin{equation} \elab{objective}
  \min_{\mapping, \priority} \expectation{\qoiE(\mapping, \priority)}
\end{equation}
where
\[
  \qoiE(\mapping, \priority) = \dt \sum \mP(\mapping, \priority),
\]
$\dt$ is the sampling interval of the power profile $\mP$, and $\sum \mP$ denotes the summation over all elements of $\mP$.
Hereafter, we also emphasize the dependency on $\mapping$ and $\priority$.
Our constraints are (i) time, (ii) temperature, and (iii) reliability as follows.
(i) The end-to-end delay of the application is constrained by $\t_\maximal$ (a deadline).
(ii) The maximal temperature that the system can tolerate is constrained by $\q_\maximal$, and $\pr_\burn$ is an acceptable probability of burning the chip.
(iii) The minimal time that the system should survive is constrained by $\T_\minimal$, and $\pr_\wear$ is an acceptable probability of having a premature fault due to wear.
The three constraints are formalized as follows:
\begin{align}
  & \period(\mapping, \priority) \leq \t_\maximal, \elab{timing-constraint} \\
  & \probabilityMeasure\left( \qoiQ(\mapping, \priority) \geq \q_\maximal \right) \leq \pr_\burn, \text{ and} \elab{thermal-constraint} \\
  & \probabilityMeasure\left( \qoiT(\mapping, \priority) \leq \T_\minimal \right) \leq \pr_\wear. \elab{reliability-constraint}
\end{align}
In \eref{timing-constraint}--\eref{reliability-constraint}, $\period$ is the end-to-end delay of the application according to the schedule,
\begin{align*}
  & \qoiQ(\mapping, \priority) = \norm[\infty]{\mQ(\mapping, \priority)}, \\
  & \qoiT(\mapping, \priority) = \expectation{\T(\mapping, \priority) \, | \, \eta} = \eta(\mapping, \priority) \, \Gamma\left(1 + \frac{1}{\beta}\right), \text{ and}
\end{align*}
$\norm[\infty]{\mQ}$ denotes the extraction of the maximal value from the temperature profile $\mQ$.
The last two constraints, \ie, \eref{thermal-constraint} and \eref{reliability-constraint}, are probabilistic as the quantities under consideration are random.
Due to the nested structure of the reliability model described in \rref{two-level-probabilistic-modeling}, the constraint in \eref{reliability-constraint} involving $\T$ requires an extra care.
For illustration, we set an upper bound on the probability of the expected value of $\T$.
It should be noted, though, that this expectation is a \rv\ itself.

In order to evaluate \eref{objective}--\eref{reliability-constraint}, we utilize the uncertainty analysis technique presented in \sref{uncertainty-analysis}.
In this case, the quantity of interest is a vector with three elements:
\begin{equation} \elab{quantity-of-interest}
  \vw = (\qoiE, \qoiQ, \qoiT).
\end{equation}
Although it is not spelled out, each quantity depends on $\mapping$ and $\priority$.
The first element corresponds to the energy consumption used in \eref{objective}, the second element is the maximal temperature used in \eref{thermal-constraint}, and the last one is the scale parameter of the reliability model (see \sref{reliability-analysis}) used in \eref{reliability-constraint}.
The uncertainty analysis in \sref{uncertainty-analysis} should be applied as explained in \rref{multiple-dimensions}.
In \aref{surrogate-construction}, Algorithm~X is an intermediate procedure that makes a call to \aref{temperature-solution} and processes the resulting power and temperature profiles as required by \eref{quantity-of-interest}.
