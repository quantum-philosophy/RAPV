Our work is motivated by the following two observations.

First, temperature is the driving force of the dominant failure mechanisms.
The most prominent examples include electromigration, time-dependent dielectric breakdown, stress migration, and thermal cycling \cite{xiang2010}; see \cite{jedec2011} for an exhaustive overview.
All of the aforementioned mechanisms have exponentially dependencies on the operating temperature.
At the same time, temperature is tightly related to process parameters, such as the effective channel length and gate-oxide thickness, and can vary dramatically when those parameters deviate from their nominal values \cite{ukhov2014, juan2012}.
Meanwhile, the state-of-the-art techniques for reliability analysis of electronic systems lack a systematic treatment of process variation and, in particular, of the effect of process variation on temperature.

Second, having determined a probabilistic model $\survival_\T(\cdot, \vwr)$ of the considered system, the major portion of the associated computational time is ascribed to the evaluation of the parametrization $\vwr$ rather than to the model \perse, that is, when $\vwr$ is known.
For instance, $\vwr$ often contains estimates of the mean time to failure of each processing element given for a range of stress levels.
Therefore, $\vwr$ typically involves (computationally intensive) full-system simulations including power analysis paired with temperature analysis \cite{xiang2010}.

Guided by the two observations mentioned earlier, we propose to the use of the spectral decompositions developed in \sref{uncertainty-analysis} and \sref{temperature-analysis} in order to construct a light surrogate for $\vwr$.
The proposed technique is founded on the basis of the state-of-the-art reliability models by enriching their modeling capabilities with respect to process variation and by speeding up the associated computational process.
This approach allows one to seamlessly incorporate into reliability analysis the effect of process variation on process parameters.
In particular, the framework allows for a straightforward propagation of the uncertainty from process parameters through power and temperature to the lifetime of the system.
In contrast to the straightforward use of MC sampling, the spectral representation that we construct makes the subsequent analysis highly efficient from the computational perspective.

It is worth noting that $\survival_\T(\cdot, \vwr)$ is left intact, meaning that our approach does not impose any restrictions on $\survival_\T(\cdot, \vwr)$ and, thus, is readily applicable to any model that the user considers to be the most adequate for the problem at hand.
