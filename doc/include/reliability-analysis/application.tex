Let us now apply our general technique to address one of the major concerns of the designer of electronic systems: the thermal-cycling fatigue \cite{jedec}.
This fatigue has a sophisticated dependency on temperature: apart from average/maximal temperatures, the frequencies and amplitudes of temperature fluctuations matter in this case.
Suppose that the system at hand is experiencing a periodic workload due to the execution of a periodic or nearly periodic application with period $\period$.
The power consumption is changing during the execution of the application, and, thus, the system is inevitably exposed to the damage from thermal oscillations.
The corresponding temperature profile $\mQ$ is then a \DSS\ profile, which, for a given $\vu$, can be computed using \aref{temperature-solution}.

Assume further that the structure of the reliability model is the one shown in \eref{reliability-model}.
Regarding the individual survival functions, we shall rely on Weibull distributions.
In this case,
\begin{equation} \elab{weibull-model}
  \ln \survival_{\T_i}(\t | \pvw) = -\left( \frac{\t}{\eta_i} \right)^{\beta_i}
\end{equation}
and the mean time to failure is
\begin{equation} \elab{weibull-expectation}
  \mu_i = \expectation{\T_i} = \eta_i \, \Gamma\left(1 + \frac{1}{\beta_i}\right)
\end{equation}
where $\eta_i$ and $\beta_i$ are the scale and shape parameters of the distribution, respectively, and $\Gamma$ is the Gamma function.
At this point, $\pvw = (\eta_1, \dotsc, \eta_\nprocs, \beta_1, \dotsc, \beta_\nprocs)$.

During one iteration of the application, the temperature of the $i$th processing element exhibits $\nsegments[i]$ cycles.
Each cycle generally has different characteristics and, therefore, causes different damage to the system.
This aspect is taken into account by adjusting $\eta_i$ as follows.
Let $\mQ$ be the \DSS\ temperature profile of the system under analysis and denote by $\mQ(i, :)$ the $i$th row of $\mQ$, which corresponds to the temperature curve of the $i$th processing element.
First, $\mQ(i, :)$ is analyzed using a peak-detection procedure in order to extract the extrema of this curve.
The found extrema are then fed to the rainflow counting algorithm \cite{xiang2010} for an adequate identification of thermal cycles.
Denote by $\ncycles[ij]$ the expected number of cycles to failure corresponding to the $i$th processing element and its $j$th cycle (as if it was the only cycle damaging the processing element).
$\ncycles[ij]$ is computed using the corresponding physical model of wear that can be found in \cite{ukhov2012, jedec, xiang2010}.
Let $\eta_{ij}$ and $\mu_{ij}$ be the scale parameter and expectation of the lifetime corresponding to the $i$th processing element under the stress of the $j$th cycle; the two are related as shown in \eref{weibull-expectation}.
Then \cite{ukhov2012, xiang2010}
\begin{equation} \elab{weibull-eta}
  \eta_i = \frac{\period}{\Gamma\left(1 + \frac{1}{\beta_i}\right) \sum_{j = 1}^{\nsegments[i]} \frac{1}{\ncycles[ij]}}.
\end{equation}
Note that $\eta_i$ accounts for process variation via temperature (in the above equation, $\ncycles[ij]$ is a function of temperature).

\begin{remark} \rlab{thermal-cycles}
A cycle need not be formed by adjacent extrema; cycles can overlap.
In this regard, the rainflow counting method is known to the best as it efficiently mitigates overestimation.
A cycle can be a half cycle meaning that only an upward or downward temperature swing is present in the time series, which is assumed to be taken into account in $\ncycles[ij]$.
\end{remark}

The shape parameter $\beta_i$ is known to be indifferent to temperature.
For simplicity, we also assume that $\beta_i$ does not depend on process parameters and $\beta_i = \beta$ for $i = 1, 2, \dotsc, \nprocs$.
However, we would like to emphasize that these assumptions are not a limitation of the proposed techniques.
Then it can be shown that the compositional survival function $\survival_\T(\cdot | \pvw)$ corresponds to a Weibull distribution, and the shape parameter of this distribution is $\beta$ whereas the scale parameter is given by the following equation:
\begin{equation} \elab{compound-weibull-eta}
  \eta = \left(\sum \left( \frac{1}{\eta_i} \right)^\beta \right)^{-\frac{1}{\beta}}
\end{equation}
where $\eta_i$ is as in \eref{weibull-eta}.
Consequently, the parameterization of the reliability model has boiled down to two parameters, $\eta$ and $\beta$, among which only $\eta$ is random.

Now we let the scale parameter $\eta$ be our quantity of interest $\w$ and apply the technique in \sref{uncertainty-analysis} to this quantity.
In this case, Algorithm~X in \aref{surrogate-construction} is an auxiliary function that makes a call to \aref{temperature-solution}, processes the resulting temperature profile as it was described earlier in this subsection, and returns $\eta$ computed according to the formula in \eref{compound-weibull-eta}.
Consequently, we obtain a light polynomial surrogate of the parameterization of the reliability model, which can be then studied from various perspectives.
The example for a dual-core system given at the end of \sref{temperature-solution} can be considered in this context as well with the only change that the dimensionality of the polynomial coefficients would be two here (since $\eta \in \real^\nprocs$ and $\nprocs = 2$).
