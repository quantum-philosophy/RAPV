Despite its generality, the proposed technique can be better appreciated considering a concrete example.
Suppose that the system is experiencing a periodic workload (\eg, due to the execution of a periodic or nearly periodic application) with period $\period$.
The corresponding temperature profile $\mQ$ is, therefore, a \dss\ temperature profile.
Hence, for a given $\vu$, $\mQ$ can be computed using \aref{temperature-solution}.

Assume that any fault of any processing element makes the system fail, and $\{ \T_i \}_{i = 1}^\nprocs$ are conditionally independent given the parameters gathered in $\vwr$.
In this case,
\[
  \T = \min_{i = 1}^\nprocs \T_i \hspace{1em} \text{and} \hspace{1em} \survival_\T(\t | \vwr) = \prod_{i = 1}^\nprocs \survival_{\T_i}(\t, \vwr).
\]
Regarding the individual survival functions, we shall rely on Weibull distributions.
In this case,
\begin{equation} \elab{weibull-model}
  \ln \survival_{\T_i}(\t, \vwr) = -\left( \frac{\t}{\eta_i} \right)^{\beta_i}
\end{equation}
where $\eta_i$ and $\beta_i$ are called the scale and shape parameters of the distribution, respectively.
The mean time to failure is
\begin{equation} \elab{weibull-expectation}
  \mu_i = \expectation{\T_i} = \eta_i \, \Gamma\left(1 + \frac{1}{\beta_i}\right)
\end{equation}
where $\Gamma$ is the Gamma function.
Both $\eta_i$ and $\beta_i$ depend on the considered technological process and, therefore, are affected by process variation.
Consequently, $\eta_i$ and $\beta_i$ are \rvs\ at the design stage.
The amount of stress that the $i$th processing element undergoes is time dependent as temperature changes over time.
Since the impact of temperature on $\beta_i$ is negligible, the only parameter that varies with temperature is $\eta_i$, which is taken into account as follows.

The $i$th row of $\mQ$, denoted by $\mQ(i, :)$, is the temperature curve of the $i$th processing element.
This curve is analyzed, and a set of $\nsegments[i]$ time segments $\{ \dt_{ij} \}_{j = 1}^{\nsegments[i]}$ is extracted such that the processing element undergoes a constant stress within each segment.
Let $\eta_{ij}$ and $\mu_{ij}$ be the scale parameter and expectation of the lifetime corresponding to the $i$th processing element under the stress of the $j$th cycle; the two are related as shown in \eref{weibull-expectation}.
Then the scale parameter in \eref{weibull-model} is \cite{xiang2010}
\begin{equation} \elab{compound-weibull-eta}
  \eta_i = \frac{\period}{\sum_j\frac{\dt_{ij}}{\eta_{ij}}} = \frac{\period}{\Gamma\left(1 + \frac{1}{\beta_i}\right) \sum_j\frac{\dt_{ij}}{\mu_{ij}}}
\end{equation}
The tuples $\{ (\dt_{ij}, \mu_{ij}) \}_{j = 1}^{\nsegments[i]}$, characterizing the $i$th processing element, depend on the fatigue under consideration.

Assume that the power consumption is rapidly changing during the execution of the application, and, thus, the major concern of the designer is the thermal-cycling fatigue \cite{jedec2011}.
This fatigue has a profound dependency on temperature: apart from average/maximal temperatures, the frequencies and amplitudes of temperature fluctuations matter in this case.
Each tuple $(\dt_{ij}, \mu_{ij})$ characterizes a cycle and is computed as follows.
First, $\mQ(i, :)$ is analyzed using a peak-detection procedure in order to extract the extrema of this curve.
The found extrema are then fed to the rainflow counting algorithm \cite{xiang2010} for an adequate identification of thermal cycles.
Denote by $\ncycles[ij]$ the expected number of cycles to failure corresponding to the $i$th processing element and its $j$th cycle (as if it was the only cycle damaging the processing element).
$\ncycles[ij]$ is computed using the corresponding physical model of wear given in \cite{jedec2011, xiang2010}.
Consequently, in \eref{compound-weibull-eta}, $\mu_{ij} = \ncycles[ij] \dt_{ij}$, and $\dt_{ij}$ is the duration of the cycle.
\begin{remark} \rlab{thermal-cycles}
A cycle need not be formed by adjacent extrema; cycles can overlap.
In this regard, the rainflow counting method is known to the best as it efficiently mitigates overestimation.
A cycle can be a half cycle meaning that only an upward or downward temperature swing is present in the time series, which is assumed to be taken into account in $\ncycles[ij]$.
\end{remark}
