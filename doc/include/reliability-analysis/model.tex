Let $\T: \outcomes \to \real$ be a \rv\ representing the lifetime of the considered system.
The lifetime is the time until the system experiences a fault after which the system no longer meets the requirements imposed on this system.
Let $\distribution_\T(\cdot | \pvw)$ be the distribution of $\T$ where $\pvw = (\pw_i)$ is a vector of parameters.
The survival function of the system is
\[
  \survival_\T(\t | \pvw) = 1 - \distribution_\T(\t | \pvw).
\]

The overall lifetime $\T$ is a function of the lifetimes of the processing elements, which are denoted by a set of \rvs\ $\{ \T_i \}_{i = 1}^\nprocs$.
Each $\T_i$ is characterized by a physical model of wear \cite{jedec} describing the fatigues that the corresponding processing element is suffering from.
Each $\T_i$ is also assigned an individual survival function $\survival_{\T_i}(\cdot | \pvw)$ describing the rate of failures due to those fatigues.
The structure of $\survival_\T(\cdot | \pvw)$ with respect to $\{ \survival_{\T_i}(\cdot | \pvw) \}_{i = 1}^\nprocs$ is problem specific, and it can be especially diverse in the context of fault-tolerant systems.
Thus, $\survival_\T(\cdot | \pvw)$ is to be specified by the designer of the system.

Our work in this context is motivated by the following two observations.
First, temperature is the driving force of the dominant failure mechanisms.
The most prominent examples include electromigration, time-dependent dielectric breakdown, stress migration, and thermal cycling \cite{xiang2010}; see \cite{jedec} for an exhaustive overview.
All of the aforementioned mechanisms have exponentially dependencies on the operating temperature, which is taken into account by considering the parameters in $\pvw$ as adequate functions of temperature.
At the same time, temperature is tightly related to process parameters, such as the effective channel length and gate-oxide thickness, and can vary dramatically when those parameters deviate from their nominal values \cite{ukhov2014, juan2012}.
Meanwhile, the state-of-the-art techniques for reliability analysis of electronic systems lack a systematic treatment of process variation and, in particular, of the effect of process variation on temperature.

Second, having determined a probabilistic model $\survival_\T(\cdot | \pvw)$ of the considered system, the major portion of the associated computational time is ascribed to the evaluation of the parametrization $\pvw$ rather than to the model \perse, that is, when $\pvw$ is known.
For instance, $\pvw$ often contains estimates of the mean time to failure of each processing element given for a range of stress levels.
Therefore, $\pvw$ typically involves (computationally intensive) full-system simulations including power analysis paired with temperature analysis \cite{xiang2010}.

\begin{remark} \rlab{two-level-probabilistic-modeling}
It is important to realize that there are two levels of probabilistic modeling here.
First of all, the reliability model \perse\ is a probabilistic model describing the lifetime of the system.
Second, the parametrization $\pvw$ is another probabilistic model characterizing the impact of the uncertainty due to process variation on the reliability model.
Consequently, the overall model can be thought of as a probability distribution over probability distributions.
Given an outcome of the fabrication process, that is, $\pvw$, the lifetime remains random.
\end{remark}
