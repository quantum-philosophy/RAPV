\lettrine[findent=0.4em, nindent=0em]{\textbf{P}}{rocess variation} constitutes one of the major concerns of electronic system designs \cite{srivastava2010}.
A crucial implication of process variation is that it renders the key parameters of a technological process, \eg, the effective channel length, gate oxide thickness, and threshold voltage, as random quantities at the design stage.
Therefore, the same workload applied to two ``identical'' dies can lead to two different power and, thus, temperature profiles since the dissipation of power and heat essentially depends on the aforementioned stochastic parameters.
This concern is especially urgent due to the interdependence between the leakage power and temperature \cite{liu2007}.
Consequently, process variation leads to performance degradation in the best case and to severe faults or burnt silicon in the worst scenario.
Under these circumstances, uncertainty quantification \cite{maitre2010} has evolved into an indispensable asset of temperature-aware design workflows in order to provide them with guaranties on the efficiency and robustness of products.

\Ta\ can be broadly classified into two categories: transient and steady-state.
The latter can be further subdivided into static and dynamic.
\Tta\ is concerned with studying the thermal behavior of a system as a function of time.
Intuitively speaking, the analysis takes a power curve and delivers the corresponding temperature curve.
\Sssta\ addresses the hypothetical scenario in which the power dissipation is constant, and one is interested in the temperature that the system will attain when it reaches a static steady state.
In this case, the analysis takes a single value for power (or a power curve which is immediately averaged out) and outputs the corresponding single value for temperature.
\Dss\ (\DSS) \ta\ is a combination of the previous two: it is also targeted at a steady state of the system, but this steady state, referred to as a dynamic steady state, is now a temperature curve rather than a single value.
The considered scenario is that the system is exposed to a periodic workload or to such a workload that can be approximated as periodic, and one is interested in the repetitive evolution of temperature over time when the thermal behavior of the system stabilizes and starts exhibiting the same pattern over and over again.
Prominent examples here are various multimedia applications.
The input to the analysis is a power curve, and the output is the corresponding periodic temperature curve.
In the absence of uncertainty, this type of analysis can be efficiently undertaken using the technique developed in \cite{ukhov2012}.

A typical design task, for which \ta\ is of central importance, is temperature-aware reliability analysis and optimization.
The crucial impact of temperature on the lifetime of electronic circuits is well known \cite{jedec}.
Examples of the commonly considered failure mechanisms include electromigration, time-dependent dielectric breakdown, and thermal cycling, which are directly driven by temperature.
Among all failure mechanisms, thermal cycling has arguably the most prominent dependence on temperature: not only the average and maximum temperature but also the amplitude and frequency of temperature oscillations have a huge impact on the lifetime of the circuit.
In this context, the availability of detailed temperature profiles is essential, which can be delivered by means of either transient or \DSS\ \ta.

Due to the urgent concern originating from process variation, deterministic temperature analysis and, thus, all procedures based on it are no longer a viable option for the designer.
The presence of uncertainty has to be addressed in order to pursue efficiency and fail-safeness.
In this context, probabilistic techniques are the way to go, which, however, implies a higher level of complexity.
This paper builds upon the state-of-the-art techniques for deterministic \DSS\ \ta\ proposed in \cite{ukhov2012} and probabilistic \tta\ proposed in \cite{ukhov2014} and presents a computationally efficient framework for probabilistic \DSS\ \ta\ and the subsequent reliability analysis and optimization of electronic systems.

The remainder of the paper is organized as follows.
\sref{prior-work} provides an overview of the prior work.
In \sref{present-work}, we summarize the contribution of the present paper.
The objective of our study is formulated in \sref{problem-formulation}.
The proposed frameworks for uncertainty, temperature, and reliability analyses are presented in \sref{uncertainty-analysis}, \sref{temperature-analysis}, and \sref{reliability-analysis}, respectively.
An application of the proposed techniques in the context of reliability optimization is given in \sref{reliability-optimization}.
The experimental results are reported and discussed in \sref{experimental-results}.
\sref{conclusion} concludes the paper.
The appendix contains a set of supplementary materials with discussions on certain aspects of our solutions.
