Process variation constitutes one of the major concerns of electronic system designs \cite{chandrakasan2001, srivastava2010}.
A crucial implication of process variation is that it renders the key parameters of a technological process, \eg, the effective channel length, gate oxide thickness, and threshold voltage, as random quantities at the design stage.
Therefore, the same workload applied to two ``identical'' dies can lead to two different power and, thus, temperature profiles since the dissipation of power and heat essentially depends on the aforementioned stochastic parameters.
This concern is especially urgent due to the interdependence between the leakage power and temperature \cite{srivastava2010}.
Consequently, process variation leads to performance degradation in the best case and to severe faults or burnt silicon in the worst scenario.
Under these circumstances, uncertainty quantification (UQ) \cite{xiu2010, maitre2010} has evolved into an indispensable asset of electronic system design workflows in order to provide them with guaranties on the efficiency and robustness of products.
