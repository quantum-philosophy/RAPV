As shown in \xeref{orthogonal-projection}, each $\coefficient{\w}_{\multiindex}$ is an $\nvars$-dimensional integral, which, in general, should be computed numerically.
This task is accomplished by virtue of an adequate $\nvars$-dimensional quadrature, which is essentially a set of $\nvars$-dimensional points accompanied by scalar weights.
Since we are interested in integration with respect to the standard Gaussian measure over $\real^\nvars$ (see \xeref{standard-gaussian-measure}), we shall rely on the Gauss-Hermite family of quadrature rules \cite{maitre2010}, which is a subset of a broader family known as Gaussian quadratures.
The construction of high-dimensional rules should be undertaken with a great care as, without special treatments, the number of points grows exponentially.
In what follows, we address this crucial aspect.

Let $f: \real^\nvars \to \real$ and define the quadrature-based approximation of the integral of $f$ by the linear functional
\[
  \quadrature{\nvars}{\nqlevel}{f} := \sum_{i = 1}^\nqorder f(\point_i) \, \weight_i
\]
where $\{ (\point_i \in \real^\nvars, \weight_i \in \real) \}_{i = 1}^\nqorder$ are the points and weights of the chosen quadrature.
\rref{elementwise-operations} applies in this context as well.
The subscript $\nqlevel \in \natural$ denotes the level of the rule, which is its index in the corresponding family of rules with increasing precision.
The precision refers to the maximal total order of polynomials that the quadrature integrates exactly.
The number of points $\nqorder$ can be deduced from the pair $(\nvars, \nqlevel)$, which we shall occasionally emphasize by writing $\nqorder(\nvars, \nqlevel)$.
For the Gauss-Hermite quadrature rules in one dimension, we have that $\nqorder = \nqlevel + 1$ and the precision is $2 \nqorder - 1$ or, equivalently, $2 \nqlevel + 1$, which is a remarkable property of Gaussian quadratures.

The foundation of a multidimensional rule $\quadrature$ is a set of one-dimensional counterparts $\{ \quadrature{1}{i} \}_{i = 0}^\nqlevel$.
A straightforward construction is the tensor product of $\nvars$ copies of $\quadrature{1}{\nqlevel}$:
\begin{equation} \xelab{full-tensor-product-grid}
  \quadrature = \bigotimes_{i = 1}^\nvars \quadrature{1}{\nqlevel},
\end{equation}
which is referred to as the full-tensor product.
In this case, $\nqorder(\nvars, \nqlevel) = \nqorder(1, \nqlevel)^\nvars$, \ie, the growth of the number of points is exponential.
Moreover, it can be shown that most of the points obtained via this construction are an excess as the full-tensor product does not take into account the fact that the integrands under consideration are polynomials whose total order is constrained according to a certain strategy.

An alternative construction is the Smolyak algorithm \cite{maitre2010, eldred2008}.
Intuitively, the algorithm combines $\{ \quadrature{1}{i} \}_{i = 0}^\nqlevel$ such that $\quadrature$ is tailored to be exact only for a specific polynomial subspace.
Define $\Delta_0 := \quadrature{1}{0}$ and $\Delta_i := \quadrature{1}{i} - \quadrature{1}{i - 1}$ for $i \geq 1$.
Then Smolyak's approximating formula is
\begin{equation} \xelab{smolyak-sparse-grid}
  \quadrature = \bigoplus_{\multiindex \in \quadratureMultiindexSet{\nqlevel}} \Delta_{\alpha_1} \otimes \cdots \otimes \Delta_{\alpha_\nvars},
\end{equation}
where $\multiindex$ is a multi-index, and $\quadratureMultiindexSet{\nqlevel}$ is as defined in \xeref{total-order-index-set}.
It can be seen that the construction is a summation of cherry-picked tensor products of one-dimensional quadrature rules.
\xeref{smolyak-sparse-grid} is well suited for grasping the structure of the resulting sparse grids; more implementation-oriented versions of the Smolyak formula can be found in the cited literature.
\begin{remark}
Although we use the same notation in \xeref{full-tensor-product-grid} and \xeref{smolyak-sparse-grid}, it should be understood that the two constructions are generally different.
The latter reduces to the former if $\quadratureMultiindexSet{\nqlevel}$ is set to $\{ \multiindex: \max \multiindex \leq \nqlevel \}$ with $\max \multiindex := \max_i \alpha_i$.
\end{remark}
