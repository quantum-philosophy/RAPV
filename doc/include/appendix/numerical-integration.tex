As shown in \xeref{orthogonal-projection}, each $\coefficient{\w}_{\multiindex}$ is an $\nvars$-dimensional integral, which, in general, should be computed numerically.
This task is accomplished by virtue of an adequate $\nvars$-dimensional quadrature rule \cite{press2007}, which is essentially a set of $\nvars$-dimensional points accompanied by scalar weights.
Since we are interested in integration with respect to the standard Gaussian measure over $\real^\nvars$ (see \xeref{standard-gaussian-measure}), we shall rely on the Gauss-Hermite family of rules, which is a subset of a broader category known as Gaussian quadratures.
The construction of high-dimensional quadratures should be undertaken with a great care since a liner increase of $\nvars$ can lead to an exponential growth of the number of points.
In what follows, we address this crucial aspect.

Let $f: \real^\nvars \to \real$ be a function to integrate and $\{ (\point_i \in \real^\nvars, \weight_i \in \real) \}_{i = 1}^\nqorder$ be the points and weights of the chosen quadrature.
The corresponding approximation of the integral is defined by the following linear functional:
\[
  \quadrature{\nvars}{\nqlevel}{f} := \sum_{i = 1}^\nqorder f(\point_i) \, \weight_i \approx \int_{\real^\nvars} f(\point) \, \d \point
\]
where $\nqlevel \in \natural$ denotes the accuracy level, which is the index of the rule in the corresponding family of rules with increasing precision.
The precision typically refers to the maximal total order of polynomials that the rule integrates exactly.
The number of points $\nqorder$ that the rule has can be deduced from the pair $(\nvars, \nqlevel)$, which we shall occasionally emphasize by writing $\nqorder(\nvars, \nqlevel)$.
For the chosen quadrature in one dimension, we have that $\nqorder = \nqlevel + 1$ and the corresponding precision is $2 \nqorder - 1$ or, equivalently, $2 \nqlevel + 1$, which is a remarkable property of Gaussian quadratures.

The foundation of a multidimensional rule $\quadrature$ is a set of one-dimensional ones $\{ \quadrature{1}{i} \}_{i = 0}^\nqlevel$.
A straightforward technique to construct $\quadrature$ is the tensor product of $\nvars$ copies of $\quadrature{1}{\nqlevel}$:
\[
  \quadrature = \bigotimes_{i = 1}^\nvars \quadrature{1}{\nqlevel},
\]
which is referred to as the full-tensor product construction.
However, in this case, $\nqorder(\nvars, \nqlevel) = \nqorder(1, \nqlevel)^\nvars$, \ie, the growth of the number of points is exponential.
Moreover, it can be shown that the vast majority of points obtained via this approach do not contribute to the asymptotic accuracy.
Recall that our approximating functions are polynomials whose total order is constrained according to a certain strategy.

The integral in \xeref{orthogonal-projection} is then approximated by the sum of the integrand values evaluated at the quadrature points and multiplied by the corresponding weights:
\begin{equation} \xelab{non-intrusive-orthogonal-projection}
  \coefficient{\w}_{\multiindex} = \sum_{i = 1}^\nqorder \w(\transformation{\point_i}) \, \polynomial_{\multiindex}(\point_i) \, \weight_i.
\end{equation}
