Let $\hilbertSpace \subset \L{2}(\outcomes, \sigmaAlgebra, \probabilityMeasure)$ be the Gaussian Hilbert space \cite{janson1997} spanned by the \rvs\ $\{ \z_i \}_{i = 1}^\nvars$ contained in $\vz$ as defined in \eref{independent-random-variables}.
Since these variables are independent and standard, they form an orthonormal basis in $\hilbertSpace$, and the dimensionality of $\hilbertSpace$ is $\nvars$.
Let $\polynomialSpace_n(\hilbertSpace)$ be the space of $\nvars$-variate polynomials over $\hilbertSpace$ such that the total order of each polynomial is less or equal to $n$.
$\polynomialSpace_n(\hilbertSpace)$ can be constructed as a span of $\nvars$-variate Hermite polynomials \cite{maitre2010, eldred2008}:
\[
  \polynomialSpace_{n}(\hilbertSpace) = \hull{\left\{ \polynomial_{\multiindex}(\vh): \norm[1]{\multiindex} \leq n, \vh = (\h_i) \in \hilbertSpace^\nvars \right\}}
\]
where $\multiindex = (\alpha_i) \in \natural^\nvars$ is a multi-index with $\norm[1]{\multiindex} := \sum_i \alpha_i$,
\[
  \polynomial_{\multiindex}(\vh) = \prod_{i = 1}^\nvars \polynomial_{\alpha_i}(\h_i), \hspace{1em} \text{and}
\]
$\polynomial_{\alpha_i}$ is a one-dimensional Hermite polynomial of order $\alpha_i$, which is assumed be normalized for convenience.
Define $\chaosSpace_0 := \polynomialSpace_0(\hilbertSpace)$ (the space of constants) and, for $n \geq 1$,
\[
  \chaosSpace_n := \polynomialSpace_n(\hilbertSpace) \, \cap \, \polynomialSpace_{n - 1}(\hilbertSpace)^\perp.
\]
The spaces $\chaosSpace_n$, $i \geq 0$, are mutually orthogonal, closed subspaces of $\L{2}(\outcomes, \sigmaAlgebra, \probabilityMeasure)$.
Since our scope of interest is restricted to functions of $\vz$ (via \eref{probability-transformation}), $\sigmaAlgebra$ is assumed to be generated by $\{ \z_i \}_{i = 1}^\nvars$.
Then, by the Cameron-Martin theorem,
\[
  \L{2}(\outcomes, \sigmaAlgebra, \probabilityMeasure) = \bigoplus_{n = 0}^\infty \chaosSpace_n,
\]
which is known as the Wiener chaos decomposition.
Thus, any $\w \in \L{2}(\outcomes, \sigmaAlgebra, \probabilityMeasure)$ admits an expansion with respect to the polynomial basis.
Define the associated linear operator by
\begin{equation} \xelab{spectral-decomposition}
  \chaos{\nvars}{\nclevel}{\w} := \sum_{\multiindex \in \chaosMultiindexSet{\nclevel}} \innerProduct{\w, \polynomial_{\multiindex}} \, \polynomial_{\multiindex}(\vz)
\end{equation}
where the index set is
\begin{equation} \xelab{total-order-index-set}
  \chaosMultiindexSet{\nclevel} = \{ \multiindex: \norm[1]{\multiindex} \leq \nclevel \}.
\end{equation}
\begin{remark} \rlab{elementwise-operations}
For multidimensional quantities of interest, all the relative operations should be undertaken elementwise.
\end{remark}
The spectral decomposition in \eref{spectral-decomposition} converges in mean square to $\w$ as $\nclevel \to \infty$.
We shall refer to $\nclevel$ as the expansion level.
The cardinality of $\chaosMultiindexSet{\nclevel}$ is
\[
  \cardinality{\chaosMultiindexSet{\nclevel}} := { \nclevel + \nvars \choose \nvars }.
\]
Denote by $\d\distribution_\z$ the probability measure induced on $\real^\nvars$ by $\vz$, which is standard Gaussian given by
\begin{equation} \xelab{standard-gaussian-measure}
  \d\distribution_\z(\point) = (2 \pi)^{-\nvars/2} \, e^{-\norm[2]{\point}^2 / 2} \, \d\point.
\end{equation}
The inner product in \xeref{spectral-decomposition} performs orthogonal projections of $\w$ onto the spaces $\chaosSpace_n$, $n = 0, \dots, \nclevel$, and is given by
\begin{equation} \xelab{orthogonal-projection}
  \coefficient{\w}_{\multiindex} = \innerProduct{\w, \polynomial_{\multiindex}} = \int_{\real^{\nvars}} \w(\transformation[\point]) \, \polynomial_{\multiindex}(\point) \, \d\distribution_\z(\point).
\end{equation}
Recall that $\w$ is a function of $\vu$, and $\transformation$, defined in \eref{probability-transformation}, bridges $\vu$ with $\vz$.
Lastly, we note that $\{ \polynomial_{\multiindex}: \multiindex \in \natural^\nvars \}$ are orthonormal with respect to $\d\distribution_\z$:
\[
  \innerProduct{\polynomial_{\multiindex}, \polynomial_{\multiindex[\beta]}} = \delta_{\multiindex \multiindex[\beta]}
\]
where $\multiindex = (\alpha_i)$ and $\multiindex[\beta] = (\beta_i)$ are two arbitrary multi-indexes, $\delta_{\multiindex \multiindex[\beta]} := \prod_i \delta_{\alpha_i \beta_i}$, and $\delta_{ij}$ is the Kronecker delta.
